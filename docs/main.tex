\documentclass[a4paper, 12pt]{article}

% --- 中文支持 ---
\usepackage[UTF8, heading=true, scheme=chinese]{ctex}

% --- 页面布局 ---
\usepackage[a4paper, left=2.5cm, right=2.5cm, top=3cm, bottom=3cm]{geometry}

% --- 颜色与图形 ---
\usepackage{xcolor}
\usepackage{graphicx}
\usepackage{tikz}
\usepackage{float}

% --- 定义品牌色 (Low-Altitude Colors) ---
\definecolor{techblue}{RGB}{0, 102, 204}    % 科技蓝 (用于标题)
\definecolor{deepspace}{RGB}{20, 30, 60}    % 深空色 (用于封面背景/大标题)
\definecolor{highlightbg}{RGB}{240, 248, 255} % 浅蓝背景 (用于高亮框)

% --- 字体增强 (可选,取决于系统字体) ---
% \setmainfont{Times New Roman}
% \setCJKmainfont{SimSun}

% --- 链接与书签 ---
\usepackage[hidelinks]{hyperref}
\hypersetup{
    colorlinks=true,
    linkcolor=deepspace,
    urlcolor=techblue,
    pdftitle={低空经济发展 LEAP 指数白皮书},
    pdfauthor={您的机构名称}
}

% --- 漂亮的文本框 (用于核心观点/Key Takeaways) ---
\usepackage[most]{tcolorbox}
\newtcolorbox{insightbox}[1][]{
    colback=highlightbg,
    colframe=techblue,
    title=\textbf{#1},
    coltitle=white,
    boxrule=0.5mm,
    arc=2mm,
    fonttitle=\bfseries
}

% --- 页眉页脚 ---
\usepackage{fancyhdr}
\pagestyle{fancy}
\fancyhf{}
\lhead{\textcolor{gray}{\small 2025 低空经济 LEAP 指数白皮书}}
\rhead{\textcolor{gray}{\small 您的机构名称}}
\cfoot{\thepage}
\renewcommand{\headrulewidth}{0.5pt}

% --- 标题样式定制 ---
\usepackage{titlesec}
\titleformat{\section}
  {\color{deepspace}\normalfont\Large\bfseries}{\thesection}{1em}{}
\titleformat{\subsection}
  {\color{techblue}\normalfont\large\bfseries}{\thesubsection}{1em}{}

% ==========================================
%                 正文开始
% ==========================================
\begin{document}

% --- 封面 ---
\begin{titlepage}
    \begin{tikzpicture}[remember picture, overlay]
        % 侧边装饰条
        \fill[techblue] (current page.north west) rectangle ([xshift=2cm]current page.south west);
        % 顶部深色块
        \fill[deepspace] (current page.north west) rectangle ([yshift=-6cm]current page.north east);
    \end{tikzpicture}

    \vspace*{2cm}

    % 标题部分
    \begin{center}
        \textcolor{white}{\LARGE \textbf{2025 中国低空经济发展研究报告}} \\[0.5cm]
        \textcolor{white}{\Huge \textbf{LEAP 指数白皮书}} \\[1cm]
        \textcolor{white}{\large \textit{Low-altitude Economy Analysis of Performance}}
    \end{center}

    \vspace{6cm}

    % 报告信息
    \begin{flushright}
        \Large
        \textbf{发布机构:} [您的研究院/公司名称] \\[0.5cm]
        \textbf{发布日期:} 2025年10月 \\[0.5cm]
        \textbf{报告编号:} LA-2025-001
    \end{flushright}

    \vfill
    \begin{center}
        \small 本报告版权归发布机构所有,未经授权不得转载。
    \end{center}
\end{titlepage}

% --- 目录 ---
\newpage
\tableofcontents
\newpage

% --- 第一章:执行摘要 ---
\section{执行摘要 (Executive Summary)}

低空经济作为新质生产力的代表,正在重塑城市空间与经济结构。本报告基于\textbf{LEAP指数(Low-altitude Economy Analysis of Performance)},对全国前沿城市的低空经济发展水平进行了深度测算。

\begin{insightbox}[核心发现]
\begin{itemize}
    \item \textbf{产业爆发期已至}:2025年行业总规模突破1.5万亿。
    \item \textbf{基础设施先行}:深圳、上海等城市在“低空智联网”建设上遥遥领先。
    \item \textbf{LEAP指数显示}:第一梯队城市在“应用场景(Usage)”维度的得分普遍高于“政策支持(Policy)”维度,说明市场驱动力正在增强。
\end{itemize}
\end{insightbox}

% --- 第二章:LEAP 模型详解 ---
\section{LEAP 评价模型详解}

为了科学量化低空经济的动态发展,我们构建了包含五个维度的 LEAP 评价体系。

\subsection{模型定义}
LEAP 模型由以下五个核心维度构成,旨在捕捉“动态运行数据”的价值:

\begin{enumerate}
    \item \textbf{L (Logistics \& Infrastructure) - 基建与物流}:衡量起降点密度与物流网络覆盖。
    \item \textbf{E (Enterprise \& Ecosystem) - 企业与生态}:衡量产业链完备度。
    \item \textbf{A (Activity \& Application) - 活跃度与应用}:基于实时飞行架次数据的活跃度监测。
    \item \textbf{P (Policy \& Pilot) - 政策与试点}:政府支持力度与空域开放度。
\end{enumerate}

\subsection{测算方法}
本指数采用熵值法(Entropy Weight Method)确定客观权重...

% --- 第三章:数据分析 ---
\section{区域发展全景分析}

\subsection{综合排名 TOP 10}
根据 LEAP 指数测算,本年度综合竞争力最强的城市依次为:

\begin{table}[H]
    \centering
    \begin{tabular}{|c|c|c|c|}
    \hline
    \textbf{排名} & \textbf{城市} & \textbf{LEAP 指数} & \textbf{优势维度} \\
    \hline
    1 & 深圳 & 94.5 & 产业生态 (E) \\
    \hline
    2 & 北京 & 89.2 & 科技创新 \\
    \hline
    3 & 上海 & 88.7 & 应用场景 (A) \\
    \hline
    \end{tabular}
    \caption{2025 LEAP 指数城市排名概览}
\end{table}

\subsection{典型案例分析}
\textbf{深圳模式}:以“无人机之都”为基础,向eVTOL产业高地转型...

% --- 第四章:结论与展望 ---
\section{趋势展望}

未来三年,低空经济将呈现以下趋势:
\begin{quote}
    ``从‘隔离运行’向‘融合飞行’转变,从‘演示验证’向‘商业闭环’跨越。''
\end{quote}

\end{document}