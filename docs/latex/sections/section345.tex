\newpage
\section{发展背景:低空经济的崛起与核心特征}

\subsection{全球竞逐的战略新高地}
低空经济是指以低空空域(通常为1000米以下,可延伸至3000米)为依托,以各种有人及无人驾驶航空器飞行活动为牵引,带动相关领域融合发展的综合性经济形态~\cite{item11}。它集成了高端制造、人工智能与数字孪生等前沿技术,已成为全球新一轮科技革命与产业变革的必争之地。美国发布《先进空中交通(AAM)国家蓝图》以抢占UAM先机~\cite{item13},欧盟通过智能交通战略推动无人机物流~\cite{item14},ICAO亦加速制定全球协调框架~\cite{item1}。各国竞相布局,旨在掌控这一未来万亿级市场的战略主导权。

\subsection{中国发展的战略引擎}
在中国,低空经济被确立为``战略性新兴产业''与``新增长引擎''~\cite{item15},对于构建现代产业体系、培育新质生产力具有里程碑意义~\cite{item16}。当前,中国低空经济正从试点探索迈向规模化发展的关键期,迫切需要科学的评估体系以识别短板、优化配置,从而保障产业在``热潮''中实现高质量冷思考与稳健行进~\cite{item17}。

\subsection{产业内涵与核心特征}
低空经济并非通用航空的简单延伸,其独特属性决定了评估逻辑的革新:
\begin{itemize}
    \item \textbf{主体海量化与无人化}:无人机成为绝对主力,管理对象从``千架级''跃升至``百万级'',对空域容量与自动化管理提出极限挑战~\cite{item18}。
    \item \textbf{场景碎片化与融合化}:应用深度嵌入物流、巡检、出行等实体经济末端,呈现高度分散且跨界融合的特征~\cite{item19}。
    \item \textbf{模式动态化与网络化}:高频次、短距离、强时效的飞行活动,使其价值与风险高度依赖于实时的空域状态与网络协同~\cite{item20}。
    \item \textbf{技术迭代化与集成化}:电动化、智能化技术驱动航空器快速迭代,与5G/6G、AI深度集成,持续重塑产业形态~\cite{item4}。
\end{itemize}
这些特征表明,低空经济的真实水平无法仅凭静态指标衡量。必须对大规模、高动态的实时飞行活动进行精准度量,才能穿透表象,洞察产业的真实效能与进化动力。

\newpage
\section{核心痛点:传统评估体系面临``五大失衡''}
在产业蓬勃发展的表象下,现行依赖传统统计的评估方式与数字化、高动态的低空经济之间存在深刻错位,导致决策层深陷``数据迷雾'',面临``五大失衡'':

\begin{itemize}
    \item \textbf{``静态数据''与``动态经济''的失衡}:传统指标(如企业数、投资额)仅是周期性``快照'',无法捕捉实时运行的活跃度与效率,导致产业真实态势在数据中``失语''~\cite{item5}。
    \item \textbf{``宏观总量''与``微观结构''的失衡}:关注总体规模而忽视结构健康。无法回答``谁在飞(垄断或多元)''、``飞什么(消费或生产)''、``在哪飞(拥堵或闲置)''等关键问题,难以识别生态的真实质量~\cite{item6}。
    \item \textbf{``事后统计''与``实时决策''的失衡}:滞后的年报数据如同``看后视镜开车'',无法对空域瓶颈、安全隐患等进行前瞻预警,致使决策被动,缺乏敏捷调控能力~\cite{item21}。
    \item \textbf{``设施建设''与``运行效能''的失衡}:重``硬投入''轻``软利用''。起降点与空域资源的实际利用率往往成谜,缺乏对投资回报的持续度量,易引发资源错配与无效建设风险~\cite{item22}。
    \item \textbf{``创新潜力''与``现实融合''的失衡}:常以专利数衡量创新,却缺乏对技术在真实场景中渗透率与融合度的观测,难以判断技术转化的实际成效~\cite{item10}。
\end{itemize}

这五大失衡表明,评估范式必须发生根本性变革:从依赖周期性、静态的统计报告,转向基于连续、实时运行数据的动态诊断。



\newpage
\section{范式革新:引入动态数据驱动的领航模型}
为穿透``数据迷雾'',本蓝皮书提出全新的 低空经济发展动态评估指数体系,其核心是名为``领航''(PILOT)的五维评价模型 。这一范式革新的核心在于确立一个新基准:低空经济的价值最终通过安全、高效、大规模的飞行活动来实现。 因此,持续产生并汇聚的飞行动态数据是衡量其发展水平最直接、最客观、最及时的``金标准''~\cite{item5}。通过构建一个多维度的动态数据评价体系,我们可以将产业的``抽象潜力''转化为``具象的、可度量的运行现实''。
\subsection{核心理念}
低空经济的价值最终通过航空器的实际运行来创造与体现。每一次飞行都是一次经济活动的发生、一次生产要素的流动、一次技术能力的验证。因此,对飞行运行本身进行全维度、全链条的深度分析,是评估低空经济发展最直接、最真实、最动态的途径。本评估体系将每一次飞行任务的架次、航迹、时长、高度、速度及关联主体信息,都视为产业活力的最小单元和经济价值的直接体现。评估焦点从传统的``建造了多少''(投入),根本性转向``运行得怎样''(产出与效能)~\cite{item6}。因此,我们构建了一条清晰的``数据-指标-维度-洞察''转化管道,确保每一个评估结论都根植于可验证的数据事实。
\begin{itemize}
    \item 数据驱动:以全量、实时、真实的低空飞行动态数据为客观输入,确保评估的即时性与真实性~\cite{item24}。
\item 系统视角:从规模、结构、时空、效能、创新五个相互关联的维度解构低空经济复杂系统,避免单一指标的片面性~\cite{item25}。
\item 价值导向:不仅衡量``有没有''、``多不多'',更重点评估``好不好''、``优不优''、``新不新'',引导产业走向高质量发展~\cite{item26}。
\item 开源透明:倡导开放的方法论与可复现的计算逻辑,推动行业评估标准共建~\cite{item27}。
\end{itemize}
\subsection{体系价值}
本评估体系的根本价值在于,它通过核心的``领航''模型(PILOT)构建了一个能够系统解构低空经济复杂性的标准化评估框架,将多维度的运行活动转化为可量化、可比较、可解释的决策知识,为产业从规模化扩张向高质量发展转型提供了关键的基础设施~\cite{item6}。

对政府与监管方而言,它是``科学决策的系统工具:
\begin{itemize}
    \item 实现发展质量的精准诊断:模型超越总量统计,通过``效能与质量''、``结构与主体''等维度,精准识别区域发展在运行效率、市场健康度、应用结构等方面的真实短板,使产业扶持政策从``大水漫灌''转向``精准滴灌''~\cite{item28}。
    \item 支撑空域与资源的精细化规划:基于``时空特征''与``效能''指数形成的分析结论,能够为航路网络优化、起降场站布局提供量化的需求依据,推动基础设施投资从``按需建设''迈向``按效规划'',提升公共资源使用效益~\cite{item29}。
    \item 建立跨区域对标与协同发展的基准:统一的五维指数体系使不同城市、省份乃至国家间的低空经济发展水平具备可比性,有助于识别差距、分享最佳实践,并为区域协同规划提供共同的评估语言~\cite{item30}。
\end{itemize}

对产业与市场主体而言,它是``战略导航的量化仪表:
\begin{itemize}
    \item 提供深度市场洞察与竞争分析:企业可通过指数解构,清晰把握不同区域的场景成熟度、市场竞争格局(如集中度指数)及技术渗透趋势,为业务布局、产品定位提供超越直觉的数据支撑~\cite{item31}。
\item 确立内部运营优化的标杆:模型将行业整体运行效能予以量化呈现,为企业对标行业平均或领先水平、发现自身在任务规划、资产利用等方面的提升空间提供了客观标尺~\cite{item32}。
\item 辅助投资价值发现与风险管理:投资者可依据``创新与融合''及``结构''指数,系统性评估赛道的前沿性、生态健康度与企业梯队,从而更有效地甄别具有长期潜力的投资对象,管理投资组合风险~\cite{item33}。
\end{itemize}

对行业与研究界而言,它是``推动共识的方法论基础:
\begin{itemize}
    \item ``领航''模型(PILOT)提供了一套开源、透明、可验证的评估方法论。其系统性的维度设计和标准化的指标构建,旨在推动学术界、咨询界及行业协会就``如何科学衡量低空经济发展''形成共识,促进评估标准的统一与数据的互联互通,为构建健康的产业生态奠定认知基础~\cite{item34}。
\end{itemize}

总而言之,以``领航''模型(PILOT)为核心的基于多源动态运行数据的低空经济发展评估指数体系是连接低空经济``动态现实''与``科学决策''的桥梁。它通过将海量运行数据转化为系统的洞察,旨在终结``凭经验决策''与``用静态规划动态''的时代,推动整个产业迈向基于精准感知和数据智能的新阶段。
