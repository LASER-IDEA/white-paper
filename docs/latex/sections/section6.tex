\newpage
\section{从数据到指数的计算路径}
本指数体系的构建遵循一套标准化、可复现的计算路径,实现了从海量多源异构数据到宏观决策指数的科学转化。本章详细阐述从基础指标到维度子指数、维度指数,最终合成综合指数的完整算法逻辑与实施步骤。

\subsection{总体计算路径:四层聚合流程}
指数计算遵循自下而上的“四层聚合”逻辑,确保每一层级的数值均具有明确的物理意义与决策指向(见~\cref{fig4}):
\begin{figure}[H]
    \centering
    \includegraphics[width=0.9\textwidth]{images/图3.png}
    \caption{四层聚合流程}
    \label{fig4}
\end{figure}

\subsection{核心计算方法详述}

\subsubsection{基础指标计算}
此阶段将原始数据清洗、加工为可直接度量的基础指标值。
\begin{table}[H]
    \centering
    \small
    \begin{tabular}{|l|p{8cm}|}
\hline 输入 & 清洗后的飞行轨迹数据、飞行计划数据、航空器注册信息及用户主数据。 \\
\hline 处理 &
\textbf{时空聚合}:按预设时空单元(如日/月/年、城市/网格)对飞行事件进行统计聚合。 \newline
\textbf{业务逻辑计算}:依据指标定义执行特定运算(如去重、比率计算)。\\
\hline 输出 & 构成“维度子指数”计算基础的数十个基础指标值。 \\
\hline
\end{tabular}
    \caption{基础指标计算流程}
    \label{tab8}
\end{table}

\subsubsection{权重确定:主客观混合赋权}
为兼顾数据客观性与战略导向性,本模型采用\textbf{主客观混合赋权法}~\cite{item44}:
\begin{itemize}
    \item \textbf{客观赋权(熵权法 EWM)}:应用于\textbf{基础指标对子指数}的权重确定。依据指标数据的离散程度(熵值)自动分配权重,数据差异越大,信息量越大,权重越高,确保由数据本身的变异性驱动评价~\cite{item46}。
    \item \textbf{主观赋权(层次分析法 AHP)}:应用于\textbf{子指数对维度指数}、\textbf{维度指数对综合指数}的权重确定。通过专家打分构建判断矩阵,将产业发展阶段特征与区域战略重点转化为定量权重,体现评价的战略引导功能~\cite{item45}。
\end{itemize}

\textbf{熵权法计算核心步骤:}
假设有 $n$ 个样本和 $m$ 个指标,构建原始矩阵 $\mathbf{X}=(x_{ij})_{n \times m}$。
1. \textbf{标准化}:消除量纲差异。
   正向指标:$r_{ij} = \frac{x_{ij} - \min(x_{j})}{\max(x_{j}) - \min(x_{j})}$;
   负向指标:$r_{ij} = \frac{\max(x_{j}) - x_{ij}}{\max(x_{j}) - \min(x_{j})}$。
2. \textbf{计算熵值}:$e_j = -k \sum_{i=1}^{n} p_{ij} \ln(p_{ij})$,其中 $p_{ij} = r_{ij}/\sum r_{ij}$,$k = 1/\ln(n)$。
3. \textbf{确定权重}:$w_j = \frac{1-e_j}{\sum (1-e_j)}$。差异系数 $1-e_j$ 越大,权重越高。

\subsubsection{指数合成方法}
采用线性加权求和法进行逐层聚合,确保计算透明与结果可解释:

\textbf{第一层:维度子指数合成(基于客观权重)}
在同一维度内,利用熵权法计算基础指标权重 $w_j$,合成子指数 $S_{ikl}$(第 $i$ 样本第 $k$ 维度第 $l$ 子指数):
$$
S_{i k l}=\sum_{j=1}^{m_{k l}}\left(w_j \times r_{i j}\right) \times 100
$$

\textbf{第二层:维度指数合成(基于战略权重)}
利用AHP确定各子指数的战略权重 $W_{kl}$,合成维度指数 $D_{ik}$:
$$
D_{i k}=\sum_{l=1}^{L_k}\left(W_{k l} \times S_{i k l}\right)
$$

\textbf{第三层:综合指数合成(基于顶层权重)}
基于AHP确定的维度权重 $V_k$,合成最终的低空综合繁荣指数(LA-CPI):
$$
C I_i=\sum_{k=1}^K\left(V_k \times D_{i k}\right)
$$

\subsection{动态追踪与更新机制}
为确保指数的时效性与长期可比性,建立全生命周期的动态管理机制:
\begin{itemize}
    \item \textbf{多频次发布}:
    \begin{itemize}
        \item \textbf{实时/周/月}:针对流量、热度等高频指标,通过仪表盘实时呈现。
        \item \textbf{年度}:发布权威综合报告(LA-CPI),进行深度归因与趋势研判。
    \end{itemize}
    \item \textbf{动态基准期管理}:设定首个完整评估年度为基准期(100点)。每3-5年或遇产业重大结构性变革时,进行基准期复审与回溯调整,确保长周期趋势的可比性。
    \item \textbf{权重动态校准}:
    \begin{itemize}
        \item \textbf{客观权重}:随每期数据自动更新,实时反映指标波动。
        \item \textbf{主观权重}:每1-2年组织专家复审,依据产业阶段变化(如从基建期转向应用期)动态调整战略权重。
    \end{itemize}
    \item \textbf{版本控制}:对指标体系、算法与权重的任何调整均实行严格的版本管理,发布《变更说明》,确保指数的公信力与透明度。
\end{itemize}
这一机制使“领航”模型不仅是静态的标尺,更是能够自我进化、持续适应产业发展的动态监测系统。
