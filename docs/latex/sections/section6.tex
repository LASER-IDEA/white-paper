\newpage
\section{低空经济发展动态评估指数体系的构建}
本章将系统阐述``低空经济发展动态评估指数体系''的完整构建。低空经济发展动态评估指数体系以``领航模型(PILOT)''--即基于动态运行数据的低空经济发展五维评价模型为核心引擎,旨在建立一套以运行数据为核心、兼具系统性、诊断性与演进性的评估新范式,为精准度量产业发展质量提供标准化工具。
\subsection{构建理念与原则}
本指数体系的构建,遵循一套从理论到实践的、旨在确保其科学性与应用性的核心原则。
\begin{itemize}
    \item 系统评估原则:从“静态统计”到“动态系统评估”:超越对单一规模或设施指标的静态关注,将低空经济视为一个由技术、主体、设施、规则、市场等要素构成的动态复杂系统~\cite{item25}。评估需刻画其整体状态、内部结构、运行过程与演进方向,反映系统的协同性与健康度。
\item 价值映射原则:从“测量投入”到“度量真实产出”:坚持结果导向与效能导向~\cite{item26}。评估重心从衡量资金、政策等传统“投入端”,转向度量由飞行活动直接产生的真实“产出端”与“结果端”(如运行架次、经济效益、场景渗透率)。认为低空经济的核心价值必须通过安全、高效的飞行活动来实现~\cite{item5}。
\item 分级诊断原则:从“宏观概览”到“微观归因”:构建具备清晰层级结构的度量体系,旨在同时服务于宏观概览与微观归因~\cite{item30}。决策者可从综合得分下钻至具体维度,并最终定位到可执行的基础指标,形成“发现问题 → 诊断维度 → 追溯根源”的分析闭环。
\item 开放演进原则:从“封闭评价”到“开源框架”:在产业快速发展的初期,评估框架本身必须是开源、透明、可迭代的~\cite{item27}。方法论公开,指标与权重可随产业成熟度、技术突破和政策重点进行动态修订,使模型能伴随产业共同成长。
\end{itemize}
\subsection{理论基础与模型架构}
\subsubsection{核心理论根基}
\begin{itemize}
\item 系统动力学与复杂性科学:将低空经济界定为一个包含多重正负反馈回路的复杂适应系统~\cite{item25}。本模型旨在识别并量化影响系统行为的关键状态变量(如存量、流量、效率等),及其动态相互作用机制。
\item 产业生态系统理论:借鉴生态学概念,强调产业发展的多样性(主体与场景)、共生性(产业融合)、适应性(创新)~\cite{item35}。这直接塑造了“结构与主体”、“创新与融合”维度的设计。
\item 新经济地理学:关注经济活动在空间上的不均衡分布与集聚效应~\cite{item36}。“时空特征”维度旨在量化低空活动的空间集聚模式、流量网络与区位优势,为理解其地理格局提供理论工具。
\item 新质生产力理论:低空经济是融合前沿技术的综合性经济形态,是新质生产力的重要实践载体。本模型特别注重对技术创新、要素升级和全要素生产率提升的度量,契合高质量发展的内在要求
\item 逻辑框架法(Logical Framework Approach, LFA):这是贯穿模型架构的隐含逻辑链。“规模与增长”涵盖产出与影响,“结构与主体”描述投入与主体,“时空特征”与“效能与质量”刻画过程效率,“创新与融合”则关注驱动系统升级的转型因素~\cite{item37}。
\end{itemize}

\subsubsection{整体体系架构:四级四流驱动}
本体系采用一个层次分明、数据驱动的“四级四流”整体架构(~\cref{fig1}),将原始数据转化为决策智慧。四级结构包括:数据层:(汇集低空飞行动态数据及宏观经济、地理信息等辅助数据)、指标层(对原始数据进行清洗、统计与计算,生成可直接度量的基础指标)、指数层(通过标准化、加权聚合,形成子指数、维度指数和综合指数)、应用层(输出指数报告、可视化图表及决策建议,服务于不同用户)。四六驱动则描述了“数据流”、“计算流”、“分析流”与“决策流”在这一架构中的动态流转过程,共同形成一个从感知到行动的完整闭环~\cite{item38}。
\begin{figure}[H]
    \centering
    \includegraphics[width=0.8\linewidth]{images/图1.png}
    \caption{四级四流驱动架构}
    \label{fig1}
\end{figure}


\subsubsection{五维模型内部逻辑关系}
五个维度构成一个具有内在因果逻辑的有机整体,共同刻画低空经济系统的运行与演进规律。这五个维度相互驱动、互为因果,形成一个动态反馈系统~\cite{item39}。
\begin{itemize}
\item 规模与增长是系统的最终产出和表象,它由系统的构成和运行方式共同决定。反映了市场的总体活跃度与扩张速度,是系统运行的直观结果。
\item 结构与主体是系统的核心构成单元,多元、健康的市场生态是产业稳健发展的基石,决定了运行的质量与创新的来源。
\item 时空与分布是系统运行规律与模式的直观体现,揭示了活动在时间和空间上的分布规律,既受主体行为影响,也反过来制约效能。是资源配置效率的直观体现。
\item 效能与质量是上述三个维度相互作用下产生的系统运行效率与健康状态,衡量从资源投入到经济和社会产出的转化效率,是高质量发展的直接反映,是评价发展质量的核心。
\item 创新与融合是驱动整个系统演进和升级的外部与内部变革力量,它渗透并长期影响着其他四个维度,是系统未来潜力的关键。
\end{itemize}

\subsection{指数体系的四层结构}
本体系是一个严谨的、自上而下可分解、自下而上可聚合的四层度量金字塔。本结构的设计精髓在于,不仅提供宏观概括,更支持灵活、精准的层级化分析。综合指数、维度指数和维度子指数均可根据不同的分析目的,被独立提取与应用,服务于从战略决策到战术优化的全链条。这种“综合指数定位问题 → 维度指数界定方向 → 子指数精准归因 → 基础指标指导行动”的分析路径,是本指数体系作为高级决策工具的核心价值所在。

\begin{table}[H]
    \centering
    \small
    \begin{tabular}{|l|p{6cm}|p{6cm}|}
\hline
\rowcolor{skygradient!100}
\multicolumn{1}{|c|}{结构层} & \multicolumn{1}{|c|}{说明} & \multicolumn{1}{|c|}{作用} \\
\hline
基础指标层 & 直接从原始数据中统计或计算得出的绝对量或比率值。它们是分析的“原子”,客观描述单一事实。 & 微观溯源与行动指导:是构成所有上层指数的基石,用于追溯问题根源、设定具体绩效目标、监控运营细节,直接指导执行层行动。 \\
\hline
维度子指数层 & 维度子指数是介于基础指标与维度指数之间的中间分析层级。在单一维度内,对多个密切相关的基础指标进行聚合,形成的细分领域得分 & 精准归因与深度洞察:将宏观的维度问题分解至具体领域,实现问题的精准定位。 \\
\hline
维度指数层 & 通过对一个维度下多个相关基础指标进行标准化、加权、聚合后,生成的标准化分数。它代表了城市在“规模”、“效能”等某一特定方面的相对发展水平。五个维度分指数可以独立使用,精准定位发展的优势与短板。 & 系统诊断与优劣势识别:直观揭示地区发展的长板与短板,精准定位需优先发力的能力维度,指导差异化策略制定。 \\
\hline
综合指数层 & 将五个维度分指数再次加权聚合后得到的最终单一分数,生成低空综合繁荣总指数(LA-CPI, Composite Prosperity Index)。是对城市低空经济发展整体水平的概括性评价。 & 战略定位与宏观对标:提供最顶层的“单一得分”,实现不同城市或区域在整体发展水平上的快速比较与排名。 \\
\hline
\end{tabular}
    \caption{四层度量结构}
    \label{table1}
\end{table}



\subsection{指数体系的五维框架}
\subsubsection{五维框架定义}
我们将海量复杂的低空运行数据,聚合至五大具有明确经济学与治理理论意义的评估维度上,构建起评估体系的支柱,见~\cref{table2}和~\cref{fig3}
\begin{table}[H]
    \small
    \centering
    \begin{tabular}{|l|p{7cm}|p{6cm}|}
\hline
\rowcolor{skygradient!100}
\multicolumn{1}{|c|}{评估维度} & \multicolumn{1}{|c|}{核心作用} & \multicolumn{1}{|c|}{关键解答问题} \\
\hline
规模与增长 & 衡量低空经济活动的体量基数与扩张态势,反映市场的基本盘与景气度。 & “产业有多大?增长有多快?市场处于哪个发展阶段?” \\
\hline
结构与主体 & 剖析市场参与主体的多样性、竞争格局与生态健康度,衡量产业的稳健性与可持续性。 & “市场由谁主导?是寡头垄断还是多元繁荣?主力应用场景是什么? \\
\hline
时空与分布 & 揭示飞行活动在时间序列和地理空间上的分布规律与集聚特征,反映网络化运行水平。 & “飞行活动在何时何地发生?热点和瓶颈在哪里?有何周期性规律?” \\
\hline
效能与质量 & 评估从空域、时间、资产等资源投入到经济与社会产出的转化效率与运行品质,是发展质量的核心。 & 资源利用是否集约高效?飞行运行是否安全、规范、经济? \\
\hline
创新与融合 & 捕捉前沿技术应用、新兴商业模式及与实体经济融合渗透的深度与广度,衡量发展潜力与先进性。 & 是否采纳了新技术、新装备?是否催生了新业态?对传统产业的赋能效果如何? \\
\hline
\end{tabular}
    \caption{五维框架说明}
    \label{table2}
\end{table}
\begin{figure}[H]
    \centering
    \includegraphics[width=0.8\linewidth]{images/图2.png}
    \caption{评估维度关系结构}
    \label{fig3}
\end{figure}
		
		
		
\subsubsection{五维框架的系统动力学诠释}
本模型的五个维度并非孤立排列,而是一个存在内在逻辑联系、相互驱动、共同演化的有机整体。其系统动力学关系可阐释如下:
\begin{itemize}
    \item 规模与增长是系统的“输出表征”与“增长引擎”:它直接体现了系统当前的活动总量和扩张速度,是系统运行最直观的结果。规模的扩大能为系统带来规模经济效应,吸引更多主体进入(影响结构与主体),但也可能对空域资源造成压力(关联效能与质量)。
    \item 结构与主体是系统的“组织架构”与“活性细胞”:多元、均衡、富有竞争力的市场主体构成是系统健康与活力的基础。优良的结构能催生更高效的运营模式(提升效能)和更多元的场景创新(驱动创新与融合),从而支撑规模的可持续增长。
    \item 时空与分布是系统的“运行图谱”与“资源分布”:它揭示了系统活动在时间和空间上的具体形态。高效的时空分布意味着资源(空域、时刻)的优化配置,直接决定了效能的高低;而异常的集聚或闲置则是指引基础设施投资与空域管理优化的关键信号。
    \item 效能与质量是系统的“健康指标”与“效率核心”:它衡量系统将投入(资源、时间)转化为产出的效率。高效能是规模可持续增长的前提,也是系统竞争力的直接体现。同时,运行质量的提升(如安全性、规范性)是降低系统风险、吸引社会资本、促进创新应用普及的基石。
    \item 创新与融合是系统的“进化动力”与“价值延伸”:它代表了系统突破现状、创造新价值的能力。技术创新(新装备、新模式)直接推动效能跃升和成本下降;产业融合则开辟新的规模增长点,并可能催生新的市场主体类型,优化结构。
\end{itemize}

这五个维度构成了一个“规模驱动结构优化,结构与时空分布决定效能,效能保障规模可持续,创新引领系统跃升”的增强回路与调节回路交织的动态模型。

\subsection{体系的灵活性与可扩展性}
本体系采用模块化设计,具备高度的灵活性,以适应多样化的评估需求:
\begin{itemize}
    \item 空间尺度灵活:模型可适配国家、区域、城市、特定示范区等不同地理单元的评估,支持跨尺度对标分析。
    \item 时间粒度可选:支持按日、周、月、季度、年度不同周期输出指数与报告,满足短期监测与长期趋势研判的差异化需求。
    \item 维度聚焦定制:可根据特定用户(如监管部门、招商部门)的需求,单独发布或深化某一维度(如“效能与质量”、“创新与融合”)的指数报告。
    \item 指标动态更新:框架允许随着技术进步与产业成熟(如绿色能源、全自主飞行普及),在各维度内动态纳入新的基础指标,确保评估体系的时代性与前瞻性~\cite{item27}。
\end{itemize}
此体系的建立,不仅为评估低空经济发展提供了全新的、数据驱动的工具,更重要的是,它倡导并实践了一种以高价值运行数据为核心资产、以持续量化分析为决策支撑的产业发展治理新范式。
