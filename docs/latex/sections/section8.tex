\newpage
\section{从数据到指数的计算路径}
本指数体系的实现,遵循一套从多源基础数据到综合指数的标准化、可复现的计算路径。本章将详细阐述从基础指标到维度子指数,再到维度指数,最终合成综合指数的完整计算逻辑、核心算法与实施步骤。
\subsection{总体计算路径:四层聚合流程}
指数计算遵循自下而上的四层聚合逻辑,其核心路径如~\cref{fig4}所示:
\begin{figure}[H]
    \centering
    \includegraphics[width=0.9\textwidth]{images/图3.png}
    \caption{四层聚合流程}
    \label{fig4}
\end{figure}

\subsection{核心计算方法详述}
\subsubsection{基础指标计算}
此阶段将原始数据加工为可直接用于评估的度量值。
\begin{table}[H]
    \centering
    \small
    \begin{tabular}{|l|p{8cm}|}
\hline 输入 & 清洗后的飞行轨迹数据、飞行计划数据、航空器与用户等主数据。 \\
\hline 处理 &
时空聚合:按预设的时空单元(如日、月、年;城市、空域网格)对飞行事件进行计数、求和与求平均。 
业务逻辑计算:根据指标定义进行运算。\\
\hline 输出 & 构成"维度子指数"计算基础的数十个基础指标值。 \\
\hline
\end{tabular}
    \caption{基础指标计算流程}
    \label{tab8}
\end{table}

\subsubsection{权重确定方法:主客观结合}
为确保指数结果的客观性、公正性与科学性,权重确定是指数科学性的核心。目前,学术界与产业界的普遍共识是采用以“客观赋权法”为主,并融合专家经验的混合模式,以平衡数据驱动与战略判断~\cite{item44}。因此,本模型采用了混合赋权法。

具体而言,在需要融入战略考量的层面,如确定子指数对维度指数、以及维度指数对综合指数的权重时,采用主观赋权法——层次分析法(Analytic Hierarchy Process, AHP)。该方法通过邀请专家依据1-9标度法对要素进行两两比较,构建判断矩阵,然后运用特征根法求解权重向量并进行归一化,最后通过计算一致性比率(CR)进行检验(通常要求CR$<0.1$),从而将专家对产业发展阶段和区域战略重点的判断系统性地转化为定量权重~\cite{item45,item46}。

在更需要体现数据内在差异的层面,即维度内各基础指标对其子指数的权重确定上,则采用客观赋权法——熵权法(Entropy Weight Method, EWM)。该方法的核心逻辑是,指标在不同评估对象间的数据差异越大(熵值越小),其包含的信息量就越丰富,在评价中应被赋予更高的权重,从而实现让数据本身的变异程度来决定其重要性~\cite{item46}。

假设有 $n$ 个样本和 $m$ 个评价指标,构成原始数据矩阵 $\mathbf{X}=(x_{ij})_{n \times m}$,其中 $x_{ij}$ 表示第 $i$ 个样本在第 $j$ 个指标上的数值,$\max(x_{j})$ 和 $\min(x_{j})$ 分别是第 $j$ 个指标在所有 $n$ 个样本中的最大值和最小值。

首先,进行数据标准化。为消除不同指标量纲和正负取向的影响,对原始数据进行标准化处理,得到标准化矩阵 $\mathbf{R}=(r_{ij})_{n \times m}$。对于正向指标(效益型,越大越好):$r_{ij} = \dfrac{x_{ij} - \min(x_{j})}{\max(x_{j}) - \min(x_{j})}$。对于负向指标(成本型,越小越好):$r_{ij} = \dfrac{\max(x_{j}) - x_{ij}}{\max(x_{j}) - \min(x_{j})}$。

接着计算指标比重,将标准化值转化为比重 $p_{ij}$,可视作该样本在此指标上的“贡献概率”,形成比重矩阵 $\mathbf{P}=(p_{ij})_{n \times m}$。$p_{ij} = \dfrac{r_{ij}}{\sum_{i=1}^{n} r_{ij}}$。此步骤确保对于任一指标 $j$,所有样本的比重之和为1。

然后计算信息熵值。计算第 $j$ 个指标的熵值 $e_j$。熵值越大,表明该指标数据的差异越小,所提供的信息量也越有限。$e_j = -k \sum_{i=1}^{n} \left( p_{ij} \times \ln(p_{ij}) \right)$,其中,$k = \frac{1}{\ln(n)}$ 为常数,用于将熵值标准化到 $[0, 1]$ 区间。当某指标下所有数据完全相同时,熵值 $e_j$ 为最大值1,表示该指标未提供任何有效信息。

接着,计算差异系数。第 $j$ 个指标的差异系数 $g_j$:$g_j = 1 - e_j$。差异系数直接反映指标内数据的离散程度。熵值 $e_j$ 越小,差异系数 $g_j$ 越大,意味着该指标的区分能力越强,提供的信息量越大。

最终,将各指标的差异系数进行归一化处理,即得到第 $j$ 个指标的最终熵权 $w_j$。$w_j = \dfrac{g_j}{\sum_{j=1}^{m} g_j}$。最终得到的权重向量 $(w_1, w_2, \dots, w_m)$ 即为基于数据客观计算出的指标权重。

通过这种主客观相结合、分层次应用的赋权策略,模型旨在兼具数学严谨性与战略前瞻性。

% 为确保指数结果的客观性、公正性与科学性,权重确定是指数科学性的核心。目前,学术界与产业界的普遍共识是采用以“客观赋权法”为主,并融合专家经验的混合模式,以平衡数据驱动与战略判断~\cite{item44}。因此,本模型采用了混合赋权法。
% 具体而言,在需要融入战略考量的层面,如确定子指数对维度指数、以及维度指数对综合指数的权重时,采用主观赋权法——层次分析法(Analytic Hierarchy Process, AHP)。该方法通过邀请专家依据1-9标度法对要素进行两两比较,构建判断矩阵,然后运用特征根法求解权重向量并进行归一化,最后通过计算一致性比率(CR)进行检验(通常要求CR<0.1),从而将专家对产业发展阶段和区域战略重点的判断系统性地转化为定量权重~\cite{item45,item46}。
% 在更需要体现数据内在差异的层面,即维度内各基础指标对其子指数的权重确定上,则采用客观赋权法——熵权法(Entropy Weight Method, EWM)。该方法的核心逻辑是,指标在不同评估对象间的数据差异越大(熵值越小),其包含的信息量就越丰富,在评价中应被赋予更高的权重,从而实现让数据本身的变异程度来决定其重要性~\cite{item46}。假设有 n 个样本和 m 个评价指标,构成原始数据矩阵 X=(xij )n×m ,其中 xij  表示第 i 个样本在第 j 个指标上的数值,max(xj ) 和 min(xj ) 分别是第 j 个指标在所有 n 个样本中的最大值和最小值。首先,进行数据标准化。为消除不同指标量纲和正负取向的影响,对原始数据进行标准化处理,得到标准化矩阵 R=(rij )n×m 。对于正向指标(效益型,越大越好):rij= =xij −min(xj )/max(xj )−min(xj ),对于负向指标(成本型,越小越好):rij= =max(xj )−xij/max(xj )−min(xj )。接着计算指标比重,将标准化值转化为比重 pij ,可视作该样本在此指标上的“贡献概率”,形成比重矩阵 P=(pij )n×m 。$p_{i j}=\frac{r_{i j}}{\sum_{i=1}^n r_{i j}}$。此步骤确保对于任一指标 jj,所有样本的比重之和为1。然后计算信息熵值。计算第 j 个指标的熵值 ej 。熵值越大,表明该指标数据的差异越小,所提供的信息量也越有限。$e_j=-k \sum_{i=1}^n\left(p_{i j} \times \ln \left(p_{i j}\right)\right)$其中,k=1/ln(n) 为常数,用于将熵值标准化到 [0, 1] 区间。当某指标下所有数据完全相同时,熵值 ej 为最大值1,表示该指标未提供任何有效信息。接着,计算差异系数。第 j 个指标的差异系数 gj, ,gj =1−ej, 差异系数直接反映指标内数据的离散程度。熵值 ej  越小,差异系数 gj  越大,意味着该指标的区分能力越强,提供的信息量越大。最终,将各指标的差异系数进行归一化处理,即得到第 j 个指标的最终熵权 wj。$w_j=\frac{g_j}{\sum_{j=1}^m g_j}$最终得到的权重向量 (w1,w2,...,wm)(w1 ,w2 ,...,wm ) 即为基于数据客观计算出的指标权重。
% 通过这种主客观相结合、分层次应用的赋权策略,模型旨在兼具数学严谨性与战略前瞻性。

\subsubsection{指数合成方法}
本指数体系采用线性加权求和法进行自下而上的逐层聚合,其核心逻辑在于:每一层级的指数值,均由下一层级的指数或基础指标值乘以其对应权重后求和得出。这一方法确保了计算过程的透明度、稳健性与结果的可解释性。整体计算流程遵循分层聚合的架构,依次包含以下三个主要层级:

第一,维度子指数计算。此步骤在同一维度内展开,通过聚合多个基础指标来合成反映该维度某一侧面的子指数。例如,在“规模”维度下,可合成“交通流量”子指数。该层级的聚合路径是从基础指标到子指数。在计算第 $i$ 个样本在第 $k$ 个维度下第 $l$ 个子指数的得分时,首先需取该子指数所涵盖的 $m_{k l}$ 个基础指标的标准化值 $r_{i j}$ ,其中$r_{i j}$ 表示第 $i$ 个样本在第 $j$ 个指标上的标准化值。然后,应用熵权法(EWM)确定各基础指标的客观权重 $w_j$(满足$\sum w_j=1$ ),并通过加权求和进行计算。具体公式如下。最终,输出$S_{i k l}$ ,即第 $i$ 个样本在第 $k$ 个维度下第 $l$ 个子指数的得分,完整反映了该子指数在给定维度内的综合表现。

$$
S_{i k l}=\sum_{j=1}^{m_{k l}}\left(w_j \times r_{i j}\right) \times 100
$$


第二,维度指数计算。此步骤将同一维度下的所有子指数进一步合成,得到代表该维度整体发展水平的单一指数。例如,“规模与增长”维度的总分即由此生成。该层级的聚合路径是从子指数到维度指数。在获得第 $k$ 个维度下的所有 $L_k$ 个子指数得分 $S_{i k l}$ 后,需要进一步合成该维度的综合指数。这一步骤采用层次分析法(AHP)为各子指数分配战略权重 $W_{k l}$ (满足 $\sum W_{k l}=1$ ),以反映其在该维度内的相对重要性。随后,通过对所有子指数得分进行加权求和计算,具体公式如下,最终输出的$D_{i k}$ 即为第 $i$ 个样本在第 $k$ 个维度上的指数得分,它整合了该维度下各子指数的信息,代表了样本在此维度上的整体表现水平。

$$
D_{i k}=\sum_{l=1}^{L_k}\left(W_{k l} \times S_{i k l}\right)
$$


第三,综合指数计算。此步骤为整个体系的顶层整合,将全部五大维度指数合成为一个最终的综合指数,旨在从全局视角综合评价低空经济发展的整体水平。该层级的聚合路径是从维度指数到综合指数。在获得所有 $K$ 个(通常为 5 个)维度指数得分 $D_{i k}$ 后,最终的低空综合繁荣指数(LA-CPI)将通过加权求和的方式合成。这一步骤采用基于层次分析法(AHP)或通过战略研讨会确定战略权重 $V_k$ (满足 $\sum V_k=1$ )进行计算,该权重旨在反映特定评估背景与发展阶段的战略导向。其计算公式如下。由此计算得到的$C I_i$ ,即第 $i$ 个样本的LA-CPI得分,它综合了各维度的表现,是对低空经济繁荣程度的一个整体度量。


$$
C I_i=\sum_{k=1}^K\left(V_k \times D_{i k}\right)
$$


\subsection{数据更新与指数追踪机制}
为确保“领航”模型(PILOT)所生成指数的时效性、动态评估能力及长期可比性,特建立一套系统的数据更新、指数追踪与版本管理机制。该机制涵盖以下四个核心环节:
\begin{itemize}
    \item 多频次计算与发布周期:模型采用差异化的更新频率以满足不同决策节奏的需求。核心指数包括低空综合繁荣指数(LA-CPI)及各维度指数,按月度进行测算与发布,形成动态监测仪表盘;并按**年度发布权威综合报告,进行深度解读与趋势总结。基础指标与子指数:对于飞行流量、作业密度等高频基础指标,支持按日或周进行更新与可视化,实现对产业脉动的实时感知。原始数据:建立与空域管理、运营服务等系统的标准化数据接口,推动准实时的数据汇聚与清洗,作为计算底座。
    \item 动态基准期管理:为保证时间序列数据的可比性,设定科学的基准化处理规则。初始基准以首个完整评估年度作为固定基准期,将该年度各指标数据的平均值(或中位数)设定为标准化计算的基准值(即100点或标准化分值原点)。每3-5年对基准期进行一次复审。当产业发生结构性变化(如新技术大规模应用、空域政策重大调整)时,经论证,可启用新的固定基准期,并对历史数据进行回溯计算与衔接说明,确保长期趋势分析的连续性。
    \item 权重动态评审与优化:模型权重体系遵循“客观权重自动更新,主观权重定期评审”的原则。每1-2年组织领域专家,依据产业发展的新阶段、新战略,对子指数、维度指数的AHP判断矩阵进行系统性复审与更新,确保战略导向的与时俱进。在每次指数计算时,自动根据当期数据重新计算各基础指标的熵权。此举旨在使权重实时反映数据结构与差异度的客观变化,捕捉新兴指标的重要性上升。
    \item 全流程版本管理与透明度:为维护指数公信力,建立严格的版本控制与文档制度:对指标体系、计算公式、权重体系的任何调整,均以版本号进行标记和管理。任何版本更新均附有详细的《变更说明文档》,公开发布,明确记录调整内容、原因及对指数结果的可能影响。
\end{itemize}
通过上述机制,本指数体系不仅是一个静态的衡量工具,更演进为一个能够自我更新、持续优化、透明可信的动态监测与评估系统,为各相关方的长期决策提供稳定可靠的“数据罗盘”。


\newpage
\section{指数解读与城市画像:从数据到洞察}
\subsection{指数解读:三层诊断法}
本体系的最终价值,在于将复杂的量化结果转化为对城市低空经济发展状况的清晰洞察与生动“画像”。本章将系统阐述如何解读指数,并基于五维度的得分组合,勾勒出不同类型的城市发展肖像。解读应遵循“综合定位 → 维度诊断 → 关键指标归因”的三层路径,确保从宏观评价到微观行动的无缝衔接~\cite{item42}。
\begin{itemize}
    \item 第一层:综合定位与健康度评估。观察“综合发展指数”得分与排名,明确城市在整体发展水平上的段位(如引领者、追赶者、起步者)。观察五维指数雷达图。一个健康的产业系统未必五个维度均顶尖,但应相对均衡。若雷达图出现严重凹陷(如“效能”或“创新”维度显著偏低),则提示存在结构性短板,增长不可持续。
\item 第二层:维度诊断与优势识别。分析五维雷达图形态。均衡的形态代表健康发展,而显著的“凹陷”或“凸出”则揭示了结构性优势与短板。规模与增长:高得分意味着市场活跃,但需结合“增长动能指数”判断是内生增长还是补贴驱动。若规模大但增长乏力,可能市场已近饱和。结构与主体:重点解读“市场集中度指数”与“商业化成熟指数”。健康的生态应是“商业化程度高”且“市场集中度适中”(CR50在40\%-70\%),呈现“主体多元、场景务实”的特征。时空特征:“全时段运行指数”和“区域均衡指数”揭示运行韧性。高全时段运行与高均衡度,标志着产业已深度融入城市生产生活肌理。效能与质量:这是高质量发展的“试金石”。“单机作业效能”与“长航时任务占比”双高,说明产业技术先进、运营精细,而非“野蛮生长”。创新与融合:“夜间经济指数”和“生产/消费属性指数”是观察前沿活力的窗口。夜间活动活跃且生产属性强,代表产业正在创造增量经济价值。
\item 第三层:关键指标归因与故事构建。针对问题维度,下钻至其子指数与基础指标,定位问题根源。例如,“效能”低可归因于“单机作业效能”不足还是“长航时任务”占比过低。
\end{itemize}

\subsection{发展模式辨识:从评估到定位}
本体系的深层价值,在于通过对五个维度得分组合的研判,超越简单排名,辨识城市低空经济发展的内在驱动力与主导模式。这有助于城市认清自身在产业全局中的独特角色,避免同质化竞争,制定符合自身资源禀赋与阶段的差异化发展战略。基于五维雷达图的形态特征与关键指数的表现阈值,我们定义了四种典型的发展模式。
\subsubsection{四种典型发展模式}
\begin{table}[H]
    \centering
    \footnotesize
    \rotatebox{90}{
\begin{tabular}{|p{2cm}|p{4cm}|p{4cm}|p{4cm}|p{4cm}|}
\hline \rowcolor{skygradient!100} \multicolumn{1}{|c|}{模式类型} & \multicolumn{1}{|c|}{核心逻辑与驱动力} & \multicolumn{1}{|c|}{关键维度与指数特征} & \multicolumn{1}{|c|}{典型雷达图形状} & \multicolumn{1}{|c|}{战略画像} \\
% \hline 制造驱动型 & 从研发制造到应用牵引:以航空器整机及核心部件的研发、制造与测试为核心驱动力,飞行活动主要为产品迭代、验证及生产服务。产业规模与技术创新高度集中并依赖于头部制造企业。 & 
% "规模"与"创新"维度突出:新型号试飞、制造相关飞行活动活跃。 
% "效能"与"结构"维度相对偏弱:社会化、商业化运营不足,市场生态较单一。
%  & ``哑铃型''或"V型":在"规模"与"创新"两端凸起,而在"效能"、 "结构"中部凹陷,直观揭示"研-产"与"用-融"环节的断层。 & 产业基础雄厚,但市场应用与生态活力不足。应推动 "从技术验证飞向价值创造",促进制造优势转化为可持续的运营服务与市场生态优势。 \\
% \hline 场景深化型 & 从标杆示范到规模商用:以解决城市治理或产业升级中的特定痛点为突破口,通过打造高价值、可复制的标杆场景,实现商业模式的深度验证与持续运营,并以此牵引产业发展。 & "效能"与"融合"维度双高:在特定场景(如夜间物流、景区导航)中运行效率与融合深度显著。 "规模"维度稳健增长:由已验证的商业需求驱动。 & "左倾钻石型":在代表产出质量和融合程度的左侧维度(效能、融合)形成突出顶点,其他维度均衡支撑,形态扎实、特色鲜明。 & 以解决实际问题见长,商业化程度深。关键在于"立法、定标、拓面",将成功经验转化为标准规范,并在同类场景中规模化推广,形成集群效应。 \\
% \hline 基建引领型 & 从基础设施到生态枢纽:以前瞻性投资和建设低空飞行服务基站、智能通信网络、数据调度平台等新型基础设施为核心策略,通过提供优质公共服务吸引运营主体集聚,打造区域网络枢纽。 & "时空"维度全面领先:飞行活动在空间上分布均衡、在时间上韧性足,跨区域网络化枢纽特征明显。 & ``上凸型''或``金字塔型'':``时空''维度构成宽广的顶部平面,其他维度形成坚实基座,形态稳定,承载能力强。 & 基础设施先行,承载和吸引能力强。聚焦于推动基础设施的开放共享与互联互通,并围绕枢纽发展运维、数据、金融等现代服务业,从``通道经济''向``平台经济''升级。 \\
% \hline 生态培育型 & 从市场主体到内生繁荣:着力构建公平、开放、包容的营商与创新环境,培育大量 "专精特新"中小企业及多元化应用,产业活力呈现自下而上、百花齐放的内生增长特征。 & "结构"维度健康突出:市场主体多元化,市场集中度合理,企业梯队呈健康的"橄榄形"。各维度发展均衡:无明显短板,系统抗风险能力强。 & "饱满圆形"或 "规则五边形":五个维度得分均衡,形态圆润、稳健,是健康、可持续生态系统的直观体现。 & 市场活力自下而上,内生增长动力强。政策关键在于提供普惠性服务、维护公平竞争环境,激发微观主体创新。 \\
制造驱动型 & 从研发制造到应用牵引:以航空器整机及核心部件的研发、制造与测试为核心驱动力,飞行活动主要为产品迭代、验证及生产服务。产业规模与技术创新高度集中并依赖于头部制造企业。 & 
``规模''与``创新''维度突出:新型号试飞、制造相关飞行活动活跃。\newline
``效能''与``结构''维度相对偏弱:社会化、商业化运营不足,市场生态较单一。
 & ``哑铃型''或``V型'':在``规模''与``创新''两端凸起,而在``效能''、``结构''中部凹陷,直观揭示``研--产''与``用--融''环节的断层。 & 产业基础雄厚,但市场应用与生态活力不足。应推动``从技术验证飞向价值创造'',促进制造优势转化为可持续的运营服务与市场生态优势。 \\
\hline
场景深化型 & 从标杆示范到规模商用:以解决城市治理或产业升级中的特定痛点为突破口,通过打造高价值、可复制的标杆场景,实现商业模式的深度验证与持续运营,并以此牵引产业发展。 & ``效能''与``融合''维度双高:在特定场景(如夜间物流、景区导航)中运行效率与融合深度显著。\newline
``规模''维度稳健增长:由已验证的商业需求驱动。 & ``左倾钻石型'':在代表产出质量和融合程度的左侧维度(效能、融合)形成突出顶点,其他维度均衡支撑,形态扎实、特色鲜明。 & 以解决实际问题见长,商业化程度深。关键在于``立法、定标、拓面'',将成功经验转化为标准规范,并在同类场景中规模化推广,形成集群效应。 \\
\hline
基建引领型 & 从基础设施到生态枢纽:以前瞻性投资和建设低空飞行服务基站、智能通信网络、数据调度平台等新型基础设施为核心策略,通过提供优质公共服务吸引运营主体集聚,打造区域网络枢纽。 & ``时空''维度全面领先:飞行活动在空间上分布均衡、在时间上韧性足,跨区域网络化枢纽特征明显。 & ``上凸型''或``金字塔型'':``时空''维度构成宽广的顶部平面,其他维度形成坚实基座,形态稳定,承载能力强。 & 基础设施先行,承载和吸引能力强。聚焦于推动基础设施的开放共享与互联互通,并围绕枢纽发展运维、数据、金融等现代服务业,从``通道经济''向``平台经济''升级。 \\
\hline
生态培育型 & 从市场主体到内生繁荣:着力构建公平、开放、包容的营商与创新环境,培育大量``专精特新''中小企业及多元化应用,产业活力呈现自下而上、百花齐放的内生增长特征。 & ``结构''维度健康突出:市场主体多元化,市场集中度合理,企业梯队呈健康的``橄榄形''。\newline
各维度发展均衡:无明显短板,系统抗风险能力强。 & ``饱满圆形''或``规则五边形'':五个维度得分均衡,形态圆润、稳健,是健康、可持续生态系统的直观体现。 & 市场活力自下而上,内生增长动力强。政策关键在于提供普惠性服务、维护公平竞争环境,激发微观主体创新。 \\
\hline
\end{tabular}}
    \caption{应用模式类型}
    \label{table8}
\end{table}
% \FloatBarrier  % 确保之前的浮动体都已放置

\subsubsection{模式混合、演进与应用}
混合模式与现实复杂性:多数城市并非纯粹的单一样本,而呈现混合模式特征。例如,“基建-场景双轮驱动型”(时空与效能双高)或“制造-生态过渡型”(规模与结构均较好)。辨识时,应依据最突出的1-2个维度特征确定主导模式,并关注其次要模式特征,以全面理解发展动力。

动态演进路径:城市的发展模式会随产业生命周期和政策干预而动态演进。一个典型的成功演进路径可能是:“制造驱动型” → (通过市场培育)→ “生态培育型” → (通过基建升级)→ “基建引领型” → (通过场景挖掘)→ “场景深化型”。通过逐年对比雷达图形状的变化,可以清晰描绘出城市的这一转型升级轨迹。

对标分析与战略制定:明确自身模式后,城市的对标分析应更具针对性。制造驱动型城市,应重点对标如何拓展下游应用市场。场景深化型城市,应学习其他城市如何将单一场景扩展为产业生态。避免与不同模式的城市进行简单的“总分”对比,而应开展“模式内对标”或“路径对标”,学习同类模式先进者的特定经验。

发展模式辨识框架使本指数体系从“评估现状”的工具,升维为“诊断基因”与“预见未来”的战略罗盘。它不仅揭示了城市低空经济“现在在哪里”,更回答了“因何至此”以及“应向何处去”的根本问题,为绘制独具特色的发展蓝图提供了科学的决策依据。

\newpage
\section{低空经济发展指数动态评估模型的应用}
本体系的核心价值在于将数据转化为洞察,并将洞察转化为行动。本章旨在为政府、产业及投资机构提供一套清晰的“翻译”与“导航”手册,阐明如何将指数结果应用于核心决策场景。
\begin{itemize}
    \item[] 对政府与监管机构:
    \begin{itemize}
        \item[\bullet] 精准治理:依据维度短板(如“效能”低)配置资源、调整政策。
        \item[\bullet] 科学评估:追踪政策实施前后相关指数的变化(如出台夜间飞行政策后,“夜间经济指数”的走势),定量评估政策效能。
        \item[\bullet] 风险预警:监控“规模”快速增长而“安全合规”类指标下滑等矛盾组合,防范系统性风险。
    \end{itemize}
    \item[] 对运营与制造企业:
    \begin{itemize}
        \item[\bullet] 市场洞察:通过“结构”与“时空”维度指数,识别市场空白、竞争格局与业务热点区域。
        \item[\bullet] 运营对标:对标行业“效能”维度标杆,利用“单机作业效能”等子指数发现自身提升空间。
        \item[\bullet] 战略规划:参考城市“发展模式”,判断其产业生态是否与自身业务长板相匹配。
    \end{itemize}
    \item[] 对投资与研究机构:
    \begin{itemize}
        \item[\bullet] 赛道筛选:结合“综合指数”与“发展模式”,优先布局健康、高效或创新引领型的区域市场。
        \item[\bullet] 价值发现:利用“头部企业领航指数”等,识别在复杂高价值任务中具备真实技术壁垒的企业。
        \item[\bullet] 趋势研判:基于“生产/消费属性指数”等指标的跨年演变,把握产业转型的核心。
    \end{itemize}
\end{itemize}

\newpage
\section{局限性与未来展望}
本体系作为一套基于动态运行数据的评估模型,在提供创新视角与精准洞察的同时,也存在其固有的局限性。正视这些局限,并以此为基础描绘未来的演进方向,是保证该模型保持科学性与生命力的关键。
\begin{itemize}
    \item 数据依赖与质量瓶颈:模型的准确性和深度高度依赖核心动态数据的连续性、完整性与准确性。在大量数据清洗与处理后,若数据仍然存在接入中断、字段缺失或存在系统性偏差,将直接影响指数结果。
    \item 系统时滞与解释复杂性:低空经济作为一个复杂系统,从政策出台、投资投入到最终在飞行活动数据上产生显著效果,存在一定时滞。指数可能无法即时反映最新政策的影响,在解读短期波动时需谨慎归因。指数结果揭示的是“是什么”和“可能为什么”,但对深层因果机制的解读(例如,某一区域热度低,究竟是需求不足、空域限制还是基础设施缺失所致)仍需结合深入的实地调研与定性分析。
    \item 外部冲击与可比性调整:模型难以量化突发性外部冲击(如极端天气的持续影响、重大安全法规的陡然收紧)对产业的全面影响,这些事件可能导致指数出现难以用常规逻辑解释的波动。在进行长期时间序列比较或广泛区域对比时,需考虑空域管理规则、统计口径等基础条件的变化,并进行必要的标准化调整,这对方法论的一致性提出持续要求。
    \end{itemize}

面对上述局限与产业飞速发展的需求,“领航"模型(PILOT) 应在以下方向持续迭代,向更智能、更综合、更实时的“下一代评估系统”演进:
\begin{itemize}
    \item 数据融合与维度扩展;从“运行数据”单核驱动,迈向 “运行数据 + 产业经济数据 + 社会环境数据+地理信息数据” 多核驱动。
    \item 智能分析与预测模拟:引入机器学习、仿真模拟等技术,从“描述现状”迈向 “诊断归因”与“预测模拟” 。
    \item 实时性与颗粒度提升:从“T-1”或月度更新,向 “实时”监测与 “超细粒度”分析 演进
    \item 标准化与开源协作:推动核心指标、计算口径的行业共识,从“独家模型”走向 “开源生态” 。
\end{itemize}

\newpage
\section{结论与倡议}
本蓝皮书提出的以“领航”模型(PILOT)为核心的基于多源动态运行数据的低空经济发展评估指数体系代表了一种评估低空经济发展的新范式:从静态统计走向动态感知,从宏观报告走向微观洞察,从衡量投入走向评估真实产出与效能。 它首次系统性地将海量、高频的低空飞行动态数据,转化为一套结构化的、可直接服务于决策的评价体系。
我们倡议以开源项目的形式维护和发展此体系:
\begin{itemize}
    \item 核心开源:公开指标定义、计算逻辑与基础可视化代码。
\item  社区共建:由不同城市和机构贡献数据治理经验、分析模块和案例研究。
\item 持续校准:通过社区共识,定期审视和优化指标权重,保持体系的时代适应性。
\end{itemize}

我们诚邀全球城市管理者、运营企业、研究机构与开发者共同加入这一开源计划,用真实的数据驱动低空经济迈向更安全、更高效、更繁荣的未来。通过这一共同努力,我们旨在使该指数体系成为衡量和推动全球城市低空经济健康发展的通用语言和强大工具。我们相信,随着低空经济基础设施的日益完善和数字化水平的提升,动态运行数据将如同流动的血液,持续赋予这个新兴产业生命力与智慧。以数据驱动的领航模型,有望成为洞察这生命力脉搏、引导其健康蓬勃发展的核心工具。





