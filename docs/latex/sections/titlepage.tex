% % ==========================================
% %                 封面设计
% % ==========================================
% \begin{titlepage}
%     % 注意:TikZ 绝对定位需要编译两次才能正确显示
%     % 【关键修复】:将所有装饰(页眉、页脚、Logo)统一放在这一个 tikzpicture 中
%     \begin{tikzpicture}[remember picture, overlay]
        
%         % ------------------------------------------------
%         % 1. 全局背景 (全页浅白)
%         % ------------------------------------------------
%         \fill[white] (current page.south west) rectangle (current page.north east);
        
%         % ------------------------------------------------
%         % 2. 顶部装饰 (深空色曲线 + 科技蓝线条)
%         % ------------------------------------------------
%         \fill[deepspace] 
%             (current page.north west) -- 
%             (current page.north east) -- 
%             ([yshift=-8cm]current page.north east) .. controls 
%             ([yshift=-4cm, xshift=-4cm]current page.north east) and 
%             ([yshift=-10cm, xshift=4cm]current page.north west) .. 
%             ([yshift=-5cm]current page.north west) -- cycle;
            
%         \draw[techblue, line width=2pt] 
%             ([yshift=-5.2cm]current page.north west) .. controls 
%             ([yshift=-10.2cm, xshift=4cm]current page.north west) and 
%             ([yshift=-4.2cm, xshift=-4cm]current page.north east) .. 
%             ([yshift=-8.2cm]current page.north east);

%         % ------------------------------------------------
%         % 3. LOGO 区域 (右上角)
%         % ------------------------------------------------
%         % 想要替换 LOGO,请取消下面这行的注释,并注释掉虚线框部分
%         \node[anchor=north east, inner sep=1cm] at (current page.north east) {\includegraphics[width=3cm]{images/纯图形Logo-纯彩.png}};
        
%         % [占位符]
%         % \node[anchor=north east, inner sep=0.8cm] at (current page.north east) {
%         %     \begin{tikzpicture}
%         %         \fill[white, opacity=0.9, rounded corners] (0,0) rectangle (4, 1.5);
%         %         \node at (2, 0.75) {\textbf{\textcolor{deepspace}{LOGO 区域}}};
%         %         \draw[dashed, deepspace] (0.1,0.1) rectangle (3.9, 1.4);
%         %     \end{tikzpicture}
%         % };

%         % ------------------------------------------------
%         % 4. 装饰元素 (圆点)
%         % ------------------------------------------------
%         \fill[accentorange] ([xshift=3cm, yshift=-4cm]current page.north west) circle (0.15);
%         \fill[white] ([xshift=3.5cm, yshift=-3.5cm]current page.north west) circle (0.08);

%         % ------------------------------------------------
%         % 5. 底部页脚 (MOVED HERE) - 强制置于底部
%         % ------------------------------------------------
%         % 蓝色底条 (高度增加到 2.2cm 以防裁切)
%         \fill[techblue] (current page.south west) rectangle ([yshift=2.5cm]current page.south east);
        
%         % 页脚文字 (白色,稍微调高位置 yshift=0.9cm)
%         \node[anchor=south, white, yshift=0.5cm] at (current page.south) {
%             \begin{tabular}{c}
%                 \textbf{发布机构:} XX 研究院 \quad | \quad \textbf{联合发布:} XX 科技集团 \\
%                 \small 官方网站:www.leap-index.com \quad | \quad 联系电话:400-888-8888
%             \end{tabular}
%         };

%     \end{tikzpicture}

%     % --- 封面正文内容 ---
%     \vspace*{5cm} % 调整文字起始位置
    
%     \begin{center}
%         \textcolor{gray}{\fontsize{30}{40} \textbf{低空经济发展动态评估指数蓝皮书}} \\[0.5cm]
        
%         {
%             \fontsize{25}{35}\selectfont \textbf{\textcolor{deepspace}{基于领航模型(PILOT)的数据驱动洞察}} 
%         } \\[0.2cm]
        
%         \textcolor{techblue}{\Large \textsc{Performance} \textsc{Index} for \textsc{Low-altitude} \textsc{Operation} \&  \textsc{Technology}} \\[1.5cm]

% % PILOT-Performance Index for Low-altitude Operation & Technology
        
%         \begin{tcolorbox}[colback=skygradient, colframe=white, width=0.8\textwidth, arc=0mm, boxrule=0pt]
%             \centering \large \textcolor{deepspace}{``基于动态运行数据的五维评价体系,洞察万亿级蓝海新机遇''}
%         \end{tcolorbox}
%     \end{center}
    
%     % \vfill 不需要了,因为页脚已经固定绘制
    
% \end{titlepage}
\newgeometry{margin=0pt}

% ==========================================
%                 封面 (Cover)
% ==========================================
\begin{titlepage}
    \noindent
    \IfFileExists{images/cover.pdf}{%
        \includegraphics[width=\paperwidth, height=\paperheight]{images/cover.pdf}%
    }{%
        \IfFileExists{images/cover.png}{%
            \includegraphics[width=\paperwidth, height=\paperheight]{images/cover.png}%
        }{%
            % 备用方案...
            \begin{tikzpicture}[remember picture, overlay]
                \fill[blue] (current page.south west) rectangle (current page.north east);
                \node[white] at (current page.center) {Missing Cover File};
            \end{tikzpicture}%
        }%
    }%
\end{titlepage}
% 3. 恢复页边距,以免影响正文
\restoregeometry
