\section{引言:迈向数据驱动的低空经济新时代}

低空经济,作为以各种有人驾驶和无人驾驶航空器的低空飞行活动为牵引,辐射带动相关领域融合发展的综合性经济形态,已成为全球航空业创新发展的前沿领域与战略制高点~\cite{item1}。在中国,发展低空经济被明确视为培育新质生产力的重要抓手,是推动产业升级、构建现代化交通体系、保障社会民生的重要引擎~\cite{item2}。
在全球低空经济从技术探索迈向规模化、商业化爆发的关键转折点上,一个根本性的挑战日益凸显:我们如何超越传统的定性描述与滞后的宏观统计,实时、精准、系统地度量这一新兴产业的真实发展脉搏?产业的蓬勃活力与决策的“数据迷雾”形成了鲜明对比。当前,对低空经济发展的评估主要依赖于企业数量、投资规模、基础设施等宏观、静态的“投入端”指标~\cite{item3},或聚焦于特定场景的案例分析~\cite{item4}。尽管这些研究为理解产业基础与政策环境提供了有益框架,但它们本质上仍是周期性的“快照”,难以实时捕捉并精准度量以高动态飞行活动为核心的真实经济运行状况。这种“静态统计”与“动态经济”之间的根本矛盾,导致决策者时常面临“看不清、判不准、跟不上”的困境~\cite{item5},也使得低空空域作为一种新型生产要素的巨大潜力难以科学评估和有效释放,从而难以从“可通达”的自然资源,高效转化为“可运营”的高价值经济资源。

为破解这一挑战,亟需一套以动态运行数据为基石的新型评估范式。构建一套以动态运行数据为核心、能够科学诊断低空经济系统健康状况的评估体系,不仅是学术研究的迫切需要,更是引导产业从“投入建设”迈向“高质量运营”、实现科学决策与精准施策不可或缺的基础设施~\cite{item6}。

秉承“创新智能技术,创造伟大企业,发展数字经济”的使命,坚持“科技擎天、产业立地”的理念,粤港澳大湾区数字经济研究院(IDEA研究院)低空经济分院(LASER Institute)认识到,实现低空经济高质量发展的核心路径,在于推动低空空域完成从“可通达”的自然资源,到“可计算”的数字资源,最终迈向“可运营”的高价值经济资源的转变~\cite{item7}。这一转变不仅需要SILAS(智能融合低空系统)以修建数字“空中之路”,更在于建立一套与之匹配的“价值评估系统”。唯有对空域中的动态运行活动进行科学度量,才能精准评估“设施网、空联网、航路网、服务网”的效能,回答产业“发展得如何”的根本问题~\cite{item8}。为此,我们发布本蓝皮书,系统阐述 以“领航”模型(PILOT-\textsc{Performance} \textsc{Index} for \textsc{Low-altitude} \textsc{Operation} \&  \textsc{Technology})为核心的一套基于多源动态运行数据的低空经济发展评估指数体系。

“领航”模型(PILOT)的核心革命在于,它将评估的指针从规划蓝图转向了真实飞行。本体系系统地将大规模、实时的飞行架次、航迹、时长、高度及关联主体数据作为评估的“生命体征”,通过 “规模与增长”、“结构与主体”、“时空特征”、“效能与质量”、“创新与融合”五个核心维度,该体系构建了一个能够持续感知产业脉搏、精准诊断系统健康度的“数据驾驶舱”。这一定位使“领航”模型(PILOT)与现有主要依赖统计数据的宏观指数形成了本质区别与关键互补。本蓝皮书旨在实现三个核心目标。第一,确立“以运行定义发展”的新评估基准。我们推动评估重心从“建造了什么”坚决转向“运行得如何”,致力于用真实经济活动的密度、效率和价值来客观衡量发展水平~\cite{item9}。第二,提供一套开源、可复用的方法论公共产品。我们完整公开“领航”模型的构建逻辑、指标体系与计算路径,旨在推动行业评估标准的共识,使其成为任何城市或区域均可便捷使用的“开箱即用”型分析工具。第三,赋能从经验决策到数据智能决策的治理升级~\cite{item10}。本体系不仅提供宏观排名,更能支持微观归因,帮助决策者精准识别基础设施短板、洞察市场结构变化、评估政策干预效果,最终实现资源的精准配置与产业的可持续发展。

我们深信,数据是驱动低空经济质变的核心引擎。《低空经济发展动态评估指数蓝皮书》的发布,是我们将空域“可计算”理念在产业评估领域的深度实践,也是我们践行“发展数字经济”使命、为行业贡献的一套开源、透明、可验证的公共方法论。我们期望将前沿智能技术与真实产业需求紧密结合,以数据智能驱动决策科学化,助力低空经济这一新质生产力在安全可控的轨道上实现规模化、高质量、可持续发展。我们期望与各界同仁一道,以“领航”模型为共同工具,拨开数据迷雾,以清晰的洞察引领低空经济驶向高质量、可持续的未来。

% \begin{enumerate}
%     \item \textbf{马太效应初显}:深圳、上海等头部城市占据了全行业 60\% 以上的低空飞行量。
%     \item \textbf{场景分化}:长三角地区更侧重于跨海物流,而珠三角地区在城市空中交通(UAM)方面走得更快。
% \end{enumerate}