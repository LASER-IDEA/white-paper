\section{引言:迈向数据驱动的低空经济新时代}

低空经济,作为以各类航空器低空飞行活动为牵引的综合性经济形态,已跃升为全球航空业创新的战略制高点~\cite{item1}。在中国,它被明确定义为培育新质生产力的重要引擎,是推动产业升级与构建现代化交通体系的关键力量~\cite{item2}。

然而,在低空经济从技术验证迈向规模化商业应用的关键转折期,一个根本性的挑战日益凸显:如何超越传统定性描述与滞后的宏观统计,实时、精准地度量这一新兴产业的真实脉搏?当前评估主要依赖企业数量、投资规模等静态“投入端”指标~\cite{item3},本质上仅是周期性的“快照”,难以捕捉高频、动态的飞行活动。这种“静态统计”与“动态经济”间的错位,制造了决策层的“数据迷雾”,致使管理者常面临“看不清、判不准、跟不上”的困境~\cite{item5},也阻碍了低空空域从自然资源向高价值经济资源的转化。

破解这一困局,亟需建立以动态运行数据为基石的新型评估范式~\cite{item6}。粤港澳大湾区数字经济研究院(IDEA研究院)低空经济分院(LASER Institute)认为,低空经济高质量发展的核心路径,在于推动低空空域实现从“可通达”的自然资源,到“可计算”的数字资源,最终迈向“可运营”的高价值经济资源的根本性跨越~\cite{item7}。唯有对动态运行活动进行科学度量,才能精准评估“设施网、空联网、航路网、服务网”的真实效能~\cite{item8}。为此,我们发布本蓝皮书,正式提出以\textbf{“领航”模型(PILOT)}为核心的低空经济发展动态评估指数体系。

“领航”模型(PILOT)的核心变革在于将评估锚点从“规划蓝图”转向“真实飞行”。本体系将大规模、实时的飞行数据(架次、航迹、时长、高度等)视为产业的“生命体征”,通过“规模、结构、时空、效能、创新”五维框架,构建感知产业脉搏、诊断系统健康的“数据驾驶舱”。本蓝皮书旨在实现三大目标:
\begin{enumerate}
    \item \textbf{确立“以运行定义发展”的新基准}:推动评估重心从“建设投入”转向“运行产出”,用真实经济活动的密度与效率衡量发展水平~\cite{item9}。
    \item \textbf{提供开源、通用的方法论公共产品}:公开构建逻辑与计算路径,为各城市提供可复用、可定制的标准化分析工具。
    \item \textbf{赋能治理从经验决策向数据智能升级}:支持从宏观排名到微观归因的深度分析,辅助基础设施规划与政策效果评估,实现精准治理~\cite{item10}。
\end{enumerate}

我们深信,数据是驱动低空经济质变的核心引擎。《低空经济发展动态评估指数蓝皮书》不仅是空域“可计算”理念的深度实践,更是为行业贡献的一套透明、可验证的公共方法论。我们期望与各界携手,以“领航”模型为工具,拨开迷雾,引领低空经济驶向高质量、可持续的未来。

% \begin{enumerate}
%     \item \textbf{马太效应初显}:深圳、上海等头部城市占据了全行业 60\% 以上的低空飞行量。
%     \item \textbf{场景分化}:长三角地区更侧重于跨海物流,而珠三角地区在城市空中交通(UAM)方面走得更快。
% \end{enumerate}
