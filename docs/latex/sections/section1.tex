% ==========================================
%                 第一章:概述
% ==========================================
\section*{摘要}

% 当前,全球低空经济正从技术验证迈向规模化商业应用的关键阶段,中国将其定位为发展新质生产力的重要引擎~\cite{item1}。然而,传统评估体系依赖静态、滞后的统计数据,难以精准、实时地刻画这一新兴业态的动态演进与真实效能,导致政策制定、资源配置与产业发展之间存在“数据时滞”与“决策盲区”~\cite{item2}。

全球低空经济正处于从技术验证迈向规模化商业应用的关键转折期,中国更将其确立为发展新质生产力的核心引擎~\cite{item1}。然而,现行评估体系多依赖静态、滞后的统计数据,难以精准刻画这一新兴业态的动态演进,致使政策制定与资源配置面临``数据时滞''与``决策盲区''~\cite{item2}。

% 为应对这一核心挑战,本蓝皮书提出了基于动态运行数据的低空经济发展五维评价模型-``领航''模型(PILOT) 。区别于国内现有的、以产业规模、政策文本和基础设施等静态或周期性统计数据为主的评价体系,本体系旨在构建一套基于全量化运行数据、高度模块化、可实时演算的城市低空经济发展度量衡,将低空经济视为一个持续运行的数字物理系统,其每一次飞行活动(架次、时长、航迹、主体)都是该系统健康状况最直接的“生命体征”~\cite{item3}。通过将高时空分辨率的实时飞行数据导入“规模、结构、时空、效能、创新”五维框架,实现从微观运行到宏观态势的精准洞察,推动了评估范式从“静态统计”向“动态感知”、从“宏观描述”向“微观诊断”的根本性转变~\cite{item4}。

针对这一核心挑战,本蓝皮书提出基于动态运行数据的低空经济发展五维评价模型——\textbf{``领航''模型(PILOT)}。区别于传统以静态指标为主的评估逻辑,本模型将低空经济视为一个持续运行的数字物理系统,视每一次飞行活动(架次、时长、航迹、主体)为系统健康的``生命体征''~\cite{item3}。通过将高时空分辨率的实时数据导入``规模、结构、时空、效能、创新''五维框架,我们实现了从微观运行到宏观态势的精准洞察,推动评估范式完成从``静态统计''向``动态感知''、从``宏观描述''向``微观诊断''的根本性变革~\cite{item4}。

% 本蓝皮书详细阐释了“领航”模型的理论基础、指数体系、计算逻辑及应用场景。本模型的核心价值在于其开源性与可操作性,任何城市均可依据本蓝皮书提供的标准化方法论,结合本地数据,生成定制化的评估报告,从而为精准施策、产业规划与商业决策提供“数据驾驶舱”式的支持,为培育高质量、可持续的低空经济生态贡献方法论基础。

本蓝皮书详尽阐释了``领航''模型的理论基础、指标体系与计算逻辑。作为一套开源且高度可操作的标准化方法论,本模型旨在为各城市提供``开箱即用''的评估工具,构建辅助精准施策与商业决策的``数据驾驶舱'',为培育高质量、可持续的低空经济生态奠定坚实的基石。


% % --- Revised (Gemini 优化版) ---
% \begin{abstract}
% 全球低空经济正处于从技术验证向\textbf{规模化商业应用}转型的关键窗口期,中国明确将其确立为发展新质生产力的核心引擎\cite{ref_gov_2024}。然而,现行评估体系深受静态统计数据的滞后性掣肘,难以捕捉这一新兴业态的实时动态与真实效能,致使政策制定与资源配置陷入“数据时滞”与“决策盲区”。

% 针对上述痛点,本蓝皮书(Blue Book)首创基于动态运行数据的\textbf{低空经济发展五维评价模型——“领航”模型(PILOT)}。与侧重产业规模、政策文本等静态指标的传统体系不同,PILOT 模型旨在构建一套全量化、模块化且具备实时演算能力的城市级度量衡。该模型将低空经济重构为持续运行的“数字-物理系统”(Cyber-Physical System),将每一次飞行活动视为系统健康的直接表征,从而实现评估范式从“静态统计”向“动态感知”的根本性跃迁。
% \end{abstract}

