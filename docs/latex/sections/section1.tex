% ==========================================
%                 第一章:概述
% ==========================================
\section*{摘要}

全球低空经济正处于从技术验证迈向规模化商业应用的关键转折期,中国更将其确立为发展新质生产力的核心引擎~\cite{item1}。然而,现行评估体系多依赖静态、滞后的统计数据,难以精准刻画这一新兴业态的动态演进,致使政策制定与资源配置面临``数据时滞''与``决策盲区''~\cite{item2}。

针对这一核心挑战,本蓝皮书提出基于动态运行数据的低空经济发展五维评价模型——\textbf{``领航''模型(PILOT)}。区别于传统以静态指标为主的评估逻辑,本模型将低空经济视为一个持续运行的数字物理系统,视每一次飞行活动(架次、时长、航迹、主体)为系统健康的``生命体征''~\cite{item3}。通过将高时空分辨率的实时数据导入``规模、结构、时空、效能、创新''五维框架,我们实现了从微观运行到宏观态势的精准洞察,推动评估范式完成从``静态统计''向``动态感知''、从``宏观描述''向``微观诊断''的根本性变革~\cite{item4}。

本蓝皮书详尽阐释了``领航''模型的理论基础、指标体系与计算逻辑。作为一套开源且高度可操作的标准化方法论,本模型旨在为各城市提供``开箱即用''的评估工具,构建辅助精准施策与商业决策的``数据驾驶舱'',为培育高质量、可持续的低空经济生态奠定坚实的基石。

% \subsection{什么是 LEAP?}
% LEAP 指数是一套基于**动态运行数据**的综合评价模型:

% \begin{itemize}
%     \item \textbf{L (Low-altitude Infrastructure) - 低空基建}:包括垂直起降点(Vertiports)、5G-A 通信网覆盖率。
%     \item \textbf{E (Economy \& Ecosystem) - 经济生态}:产业链企业的聚集度与投融资活跃度。
%     \item \textbf{A (Activity Level) - 运行活跃度}:无人机日均飞行架次与空域申请通过率。
%     \item \textbf{P (Policy \& Pilot) - 政策与试点}:地方法规的完善度与空域开放面积。
% \end{itemize}

% \subsection{Logo 替换指南}
% 在封面代码的第 \textbf{55-65} 行,您可以找到 Logo 的配置。

% \begin{tcolorbox}[colback=skygradient, colframe=techblue, title=操作步骤]
%     1. 准备一张透明背景的 PNG 图片(例如 \texttt{mylogo.png})。\\
%     2. 将代码中的占位符注释掉。\\
%     3. 启用 \texttt{includegraphics} 命令。
% \end{tcolorbox}
