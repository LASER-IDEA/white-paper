% ==========================================
%                 第一章:概述
% ==========================================
\section*{摘要}

当前,全球低空经济正从技术验证迈向规模化商业应用的关键阶段,中国将其定位为发展新质生产力的重要引擎~\cite{item1}。然而,传统评估体系依赖静态、滞后的统计数据,难以精准、实时地刻画这一新兴业态的动态演进与真实效能,导致政策制定、资源配置与产业发展之间存在“数据时滞”与“决策盲区”~\cite{item2}。

为应对这一核心挑战,本蓝皮书提出了基于动态运行数据的低空经济发展五维评价模型-``领航''模型(PILOT) 。区别于国内现有的、以产业规模、政策文本和基础设施等静态或周期性统计数据为主的评价体系,本体系旨在构建一套基于全量化运行数据、高度模块化、可实时演算的城市低空经济发展度量衡,将低空经济视为一个持续运行的数字物理系统,其每一次飞行活动(架次、时长、航迹、主体)都是该系统健康状况最直接的“生命体征”~\cite{item3}。通过将高时空分辨率的实时飞行数据导入“规模、结构、时空、效能、创新”五维框架,实现从微观运行到宏观态势的精准洞察,推动了评估范式从“静态统计”向“动态感知”、从“宏观描述”向“微观诊断”的根本性转变~\cite{item4}。
本蓝皮书详细阐释了“领航”模型的理论基础、指数体系、计算逻辑及应用场景。本模型的核心价值在于其开源性与可操作性,任何城市均可依据本蓝皮书提供的标准化方法论,结合本地数据,生成定制化的评估报告,从而为精准施策、产业规划与商业决策提供“数据驾驶舱”式的支持,为培育高质量、可持续的低空经济生态贡献方法论基础。

% \subsection{什么是 LEAP?}
% LEAP 指数是一套基于**动态运行数据**的综合评价模型:

% \begin{itemize}
%     \item \textbf{L (Low-altitude Infrastructure) - 低空基建}:包括垂直起降点(Vertiports)、5G-A 通信网覆盖率。
%     \item \textbf{E (Economy \& Ecosystem) - 经济生态}:产业链企业的聚集度与投融资活跃度。
%     \item \textbf{A (Activity Level) - 运行活跃度}:无人机日均飞行架次与空域申请通过率。
%     \item \textbf{P (Policy \& Pilot) - 政策与试点}:地方法规的完善度与空域开放面积。
% \end{itemize}

% \subsection{Logo 替换指南}
% 在封面代码的第 \textbf{55-65} 行,您可以找到 Logo 的配置。

% \begin{tcolorbox}[colback=skygradient, colframe=techblue, title=操作步骤]
%     1. 准备一张透明背景的 PNG 图片(例如 \texttt{mylogo.png})。\\
%     2. 将代码中的占位符注释掉。\\
%     3. 启用 \texttt{includegraphics} 命令。
% \end{tcolorbox}