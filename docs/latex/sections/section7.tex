\section{低空经济发展指数动态评估体系详述}
本章将深入剖析指数体系的五个核心维度及其下属的具体指数。通过这五个维度的系统观测,领航模型能够将海量的、原子化的飞行活动数据,聚合成一幅反映低空经济系统运行健康度、效率与潜力的完整画像。
\subsection{维度一:规模与增长 - 产业的“脉搏”与“心电图”}
规模与增长维度旨在评估低空经济活动的绝对体量与动态趋势,是衡量产业基本盘和市场景气度的首要维度。它直接回答“产业有多大?”和“增长有多快?”这两个基础性问题,如同为产业把脉,绘制其发展的“心电图”。该维度的核心在于,不仅关注静态存量,更强调动态增量与趋势,以辨识市场是处于爆发期、稳定期还是瓶颈期。

\begin{table}[H]
    \centering
    \small
    \begin{tabular}{|p{5cm}|p{5cm}|p{5cm}|}
\hline 
\rowcolor{skygradient!100}\multicolumn{1}{|l|}{指数名称} & \multicolumn{1}{|l|}{定义} & \multicolumn{1}{|l|}{作用} \\
\hline 低空交通流量指数 \newline Low-Altitude Traffic Volume Index & 以基期(如首年)月均架次为 100 ,衡量区域低空飞行活动的绝对规模,反映整体市场容量。 & 回答``忙不忙'',是衡量产业基本盘的核心指标。直观反映城市低空经济的``繁忙程度''。 \\
\hline 低空作业强度指数 \newline Operation Intensity Index & 衡量单位空域面积或单位时间内的飞行活动密度。 & 反映空域资源的利用压力和运营紧凑程度。衡量真实的业务负荷。 \\
\hline 活跃运力规模指数 \newline Active Fleet Scale Index & 衡量实际投入运营的航空器及运营主体的数量规模。 & 反映市场供给侧的投入程度和参与广度。 \\
\hline 增长动能指数 \newline Growth Momentum Index & 衡量飞行活动规模的扩张速度和趋势强度。 & 判断市场处于爆发期、稳定期或瓶颈期。 \\
\hline
\end{tabular}
    \caption{维度一:规模与增长 说明}
    \label{table3}
\end{table}

\subsection{维度二:结构与主体 -生态的“解剖图”与“排行榜”}
结构与主体维度专注于解剖低空经济生态系统的内部构成与健康度,核心回答“谁在飞?”和“市场结构如何?”。一个健康、有韧性的产业生态不应是单一主体的独舞,而应是多元主体的共生共荣。本维度旨在超越总量规模,深入分析市场权力格局、商业化进程和技术路线的多样性,为评估产业的长期稳健性提供依据。
\begin{table}[H]
    \centering
    \small
\begin{tabular}{|p{4cm}|p{5cm}|p{5cm}|}
\hline \rowcolor{skygradient!100} \multicolumn{1}{|c|}{指数名称} & \multicolumn{1}{|c|}{定义} & \multicolumn{1}{|c|}{作用} \\
\hline 市场集中度指数 \newline
Market Concentration Index -- CR50
 & 衡量头部企业对市场飞行活动的控制力。~\cite{item40,item41} & 判断市场是垄断、寡头还是充分竞争,评估市场风险。 \\
\hline 商业化成熟指数 \newline Commercial Maturity Index & 衡量飞行活动由消费娱乐转向生产性、商业化应用的程度。 & 反映产业``自我造血''能力和真实经济价值。 \\
\hline 机型生态多元指数 \newline Aircraft Diversity Index & 衡量运营航空器型号的丰富性与均衡性。 & 反映技术路线的丰富度和供应链的韧性,避免单一依赖。 \\
\hline
\end{tabular}
    \caption{维度二:结构与主体 说明}
    \label{table4}
\end{table}

\subsection{维度三:时空与分布——运行的“三维热力图”与“潮汐图”}
时空与分布维度揭示低空飞行活动在时间和地理空间上的分布规律与模式,核心回答“何时飞?”、“在哪飞?”和“飞去哪?”。低空经济活动并非均匀发生,其在时空维度上的集聚与流动,直接映射出真实的需求热点、基础设施效能和空域资源瓶颈,是进行精细化管理和科学规划的核心依据。
\begin{table}[H]
    \centering
    \small
\begin{tabular}{|p{4cm}|p{5cm}|p{5cm}|}
\hline \rowcolor{skygradient!100} \multicolumn{1}{|c|}{指数名称} & \multicolumn{1}{|c|}{定义} & \multicolumn{1}{|c|}{作用} \\
\hline 区域发展均衡指数 \newline Regional Balance Index & 衡量飞行活动在不同行政或地理单元分布的均衡性。 & 识别发展热点与盲区,指导基础设施(如起降点)的均衡布局,辅助区域协同政策制定。 \\
\hline 全时段运行指数 \newline All-Time Operation Index & 衡量飞行活动在24小时内的分布离散度,值越高表示全天候运行越好。 & 评估基础设施支持能力和夜间经济潜力。 \\
\hline 季候稳定性指数 \newline Seasonal Stability Index & 衡量飞行活动受季节、月份影响的波动程度。 & 判断产业是全年稳定运行,还是受气候、节假日等因素影响显著。 \\
\hline 网络化枢纽指数 \newline Networked Hub Index & 衡量核心起降点或空域节点的航线连通性与流量集中度。 & 识别物流、载人网络的关键枢纽,指导基础设施投资。 \\
\hline
\end{tabular}
    \caption{维度三:时空与分布说明}
    \label{table5}
\end{table}
\subsection{维度四:效能与质量——发展的“效率仪表盘”}
效能与质量维度是评估低空经济发展从“量的增长”转向“质的提升”的关键,核心回答“飞得好不好?”。它衡量的是将空域、时间、航空器等稀缺资源转化为实际经济与社会产出的效率与品质,是产业核心竞争力和可持续发展能力的直接体现。

\begin{table}[H]
    \centering
    \small
\begin{tabular}{|p{5cm}|p{4cm}|p{5cm}|}
\hline \rowcolor{skygradient!100}  \multicolumn{1}{|c|}{指数名称} & \multicolumn{1}{|c|}{定义} & \multicolumn{1}{|c|}{作用} \\
\hline 单机作业效能指数 \newline Per-Unit Efficiency Index & 衡量平均每架航空器的年度使用频率。 & 评估任务规划的合理性和航空器的经济性。 \\
\hline 长航时任务占比指数 \newline Long-Endurance Mission Index & 衡量高价值、复杂作业任务的比重。 & 反映执行复杂、远程作业的能力和技术可靠性。 \\
\hline 广域覆盖能力指数 \newline Wide-Area Coverage Index & 衡量飞行活动所覆盖的地理范围广度。 & 评估服务网络的覆盖面和通达性。 \\
\hline 任务完成质量指数 \newline Task Completion Quality Index & 衡量飞行任务按计划执行的完整性和规范性。 & 监控运行安全与服务质量,识别系统性风险。 \\
\hline
\end{tabular}
    \caption{维度四:效能与质量 说明}
    \label{table6}
\end{table}

\subsection{维度五:创新与融合——未来的“探测雷达”}
创新与融合维度致力于捕捉低空经济的前沿动向与跨界潜能,核心回答“发展是否前沿?潜力如何?”。它衡量技术创新在实际运行中的渗透速度、新兴模式的成熟度以及与传统经济社会的融合深度,是观测产业进化方向和长期竞争力的“探测雷达”。
% \begin{landscape}
% \FloatBarrier  % 确保之前的浮动体都已放置
\begin{table}[H]
\small
    \centering
    \rotatebox{90}{
\begin{tabular}{|p{6cm}|p{6cm}|p{6cm}|}
\hline \rowcolor{skygradient!100} \multicolumn{1}{|c|}{指数名称} & \multicolumn{1}{|c|}{定义} & \multicolumn{1}{|c|}{作用} \\
\hline 城市微循环渗透指数 \newline Urban Micro-Circulation Index & 衡量航空器在楼宇、社区等城市末端场景的精细化应用程度。 & 衡量低空飞行作为城市毛细血管,连接不同行政区的网络化程度。反映低空经济与智慧城市、民生服务的融合深度。 \\
\hline 立体空域利用效能指数 \newline Stereoscopic Airspace Efficiency Index & 衡量对不同高度层进行差异化、协同化利用的效率。 & 评估空域精细化管理水平,挖掘三维空间潜力。 \\
\hline 低空经济 "生产/消费"属性指数 \newline Production-Consu mption Attribute Index & 
通过对比工作日与周末活跃度,判断产业驱动属性。 
\begin{itemize}
    \item[>] 指数>1.2:生产型(物流、巡检为主,周末休息)。 
\item[>] 指数<0.8:消费型(文旅、表演、航拍,周末热闹)。 
\item[>] 指数 $\approx 1.0$ :混合型。
\end{itemize} & 判断产业当前的主要驱动力和未来价值走向。 \\
\hline 低空夜间经济指数 \newline Low-Altitude Night Economy Index & 衡量夜间低空飞行活动的活跃度及其创造的经济社会价值。 & 挖掘低空经济的"时间增量"价值,评估全天候运行生态。 \\
\hline 头部企业``领航''指数 \newline Leading Entity Impact Index & 衡量头部企业在技术路线、运营模式和市场开拓上的引领作用。 & 衡量头部企业在高难度、高价值任务中的引领作用。 \\
\hline
\end{tabular}}
    \caption{维度五:创新与融合说明}
    \label{table7}
% \end{sidewaystable}
\end{table}
% \FloatBarrier  % 确保表格放置后才继续

\subsection{综合指数:低空综合繁荣指数}
低空综合繁荣总指数 (LA-CPI, Composite Prosperity Index)是5D-LEDEM模型的顶层集成与最终产出。它并非五个维度得分的简单平均,而是基于一套反映特定发展阶段战略导向的权重体系(例如,在规模化初期可能更侧重“规模与增长”,在成熟期则更强调“效能与质量”与“创新与融合”),对五个维度指数进行加权合成后得到的单一综合分值。
该指数旨在提供一个宏观、简洁但内涵丰富的总体评价。它能够对不同城市、区域在同一时间截面的低空经济发展整体水平进行横向排名与梯队划分,满足政府绩效考核、城市品牌宣传和投资区域筛选等高层次、快节奏的决策需求。然而,其更深层的价值在于为深度分析提供入口:一个综合得分背后的具体构成——即五个维度指数的得分组合与雷达图形状——才是解读该地区发展模式、识别优势短板、并制定精准策略的真正钥匙。因此,LA-CPI既是衡量发展结果的“总成绩单”,也是开启系统性诊断的“总开关”。
