\documentclass[a4paper, 10pt]{article}

% ==========================================
%                 导言区配置
% ==========================================

% --- 中文支持 ---
\usepackage[UTF8, heading=true, scheme=chinese]{ctex}
\usepackage{setspace}
% --- 页面布局 ---
% 增加底部边距,确保正文不会撞到页脚装饰条
\usepackage[a4paper, left=2.5cm, right=2.5cm, top=3cm, bottom=3cm]{geometry}
\setlength{\headheight}{15pt}
\usepackage{pdfpages}
% --- 颜色与图形 ---
\usepackage{xcolor}
\usepackage{graphicx}
\usepackage{tikz}
\usetikzlibrary{calc, fadings, shapes.geometric}
\usepackage{float}
\usepackage{rotating}
\usepackage{placeins}  % 提供 \FloatBarrier
% \usepackage{pdflscape}
\usepackage{float}
\usepackage[pagebackref=true,breaklinks=true,colorlinks,bookmarks=false]{hyperref}
% % \modulolinenumbers[5]
\usepackage{subcaption}
% \usepackage{longtable,booktabs,array}
\usepackage{multirow}

% \usepackage[
%     inner=3.5cm,    % 内侧(装订边)
%     outer=2.5cm,    % 外侧
%     top=2.5cm,
%     bottom=2.5cm,
%     bindingoffset=1cm,
%     asymmetric,
% ]{geometry}

% --- 文本框支持 ---
\usepackage[most]{tcolorbox}
\usepackage[table]{xcolor}
% --- 定义品牌色 ---
\definecolor{techblue}{RGB}{0, 102, 204}     % 科技蓝
\definecolor{deepspace}{RGB}{20, 30, 60}     % 深空色
\definecolor{skygradient}{RGB}{224, 242, 254} % 天空浅蓝
\definecolor{accentorange}{RGB}{245, 158, 11} % 点缀橙色
\usepackage{todonotes}
% --- 链接设置 ---
% \usepackage[hidelinks]{hyperref}
% \hypersetup{
%     colorlinks=true,
%     linkcolor=deepspace,
%     urlcolor=techblue,
%     pdftitle={低空经济发展动态评估指数蓝皮书}
% }
\usepackage[noabbrev,nameinlink,capitalise]{cleveref} % Clever references. Options: "fig. !1!" --> "!Figure 1!"
\Crefformat{figure}{#2图~#1#3}
\Crefmultiformat{figure}{图~#2#1#3}{ and~#2#1#3}{, #2#1#3}{ and~#2#1#3}
\Crefformat{table}{#2表~#1#3}
\Crefmultiformat{table}{表~#2#1#3}{ and~#2#1#3}{, #2#1#3}{ and~#2#1#3}
\hypersetup{
    colorlinks=true, % 启用颜色链接
    linkcolor=deepspace, % 内部链接颜色
    citecolor=green, % 引用颜色
    urlcolor=red, % URL 颜色
    linkbordercolor=techblue, % 内部链接边框颜色
    citebordercolor=green, % 引用边框颜色
    urlbordercolor=red, % URL 边框颜色
    pdftitle={低空经济发展动态评估指数蓝皮书}
}


% --- 页眉页脚 ---
\usepackage{fancyhdr}
\pagestyle{fancy}
\fancyhf{}
\lhead{\textcolor{gray}{\small 基于``领航''模型(PILOT)的数据驱动洞察}}
% \rhead{\textcolor{gray}{\small 2025年度报告}}
\cfoot{\thepage}
\renewcommand{\headrulewidth}{0.5pt}

% --- 标题样式 ---
\usepackage{titlesec}
\titleformat{\section}{\color{deepspace}\Large\bfseries}{\thesection}{1em}{}
\titleformat{\subsection}{\color{techblue}\large\bfseries}{\thesubsection}{1em}{}
\setstretch{1.5} % 1.5倍行距
% ==========================================
%                 正文开始
% ==========================================
\begin{document}

% % % ==========================================
% %                 封面设计
% % ==========================================
% \begin{titlepage}
%     % 注意:TikZ 绝对定位需要编译两次才能正确显示
%     % 【关键修复】:将所有装饰(页眉、页脚、Logo)统一放在这一个 tikzpicture 中
%     \begin{tikzpicture}[remember picture, overlay]
        
%         % ------------------------------------------------
%         % 1. 全局背景 (全页浅白)
%         % ------------------------------------------------
%         \fill[white] (current page.south west) rectangle (current page.north east);
        
%         % ------------------------------------------------
%         % 2. 顶部装饰 (深空色曲线 + 科技蓝线条)
%         % ------------------------------------------------
%         \fill[deepspace] 
%             (current page.north west) -- 
%             (current page.north east) -- 
%             ([yshift=-8cm]current page.north east) .. controls 
%             ([yshift=-4cm, xshift=-4cm]current page.north east) and 
%             ([yshift=-10cm, xshift=4cm]current page.north west) .. 
%             ([yshift=-5cm]current page.north west) -- cycle;
            
%         \draw[techblue, line width=2pt] 
%             ([yshift=-5.2cm]current page.north west) .. controls 
%             ([yshift=-10.2cm, xshift=4cm]current page.north west) and 
%             ([yshift=-4.2cm, xshift=-4cm]current page.north east) .. 
%             ([yshift=-8.2cm]current page.north east);

%         % ------------------------------------------------
%         % 3. LOGO 区域 (右上角)
%         % ------------------------------------------------
%         % 想要替换 LOGO,请取消下面这行的注释,并注释掉虚线框部分
%         \node[anchor=north east, inner sep=1cm] at (current page.north east) {\includegraphics[width=3cm]{images/纯图形Logo-纯彩.png}};
        
%         % [占位符]
%         % \node[anchor=north east, inner sep=0.8cm] at (current page.north east) {
%         %     \begin{tikzpicture}
%         %         \fill[white, opacity=0.9, rounded corners] (0,0) rectangle (4, 1.5);
%         %         \node at (2, 0.75) {\textbf{\textcolor{deepspace}{LOGO 区域}}};
%         %         \draw[dashed, deepspace] (0.1,0.1) rectangle (3.9, 1.4);
%         %     \end{tikzpicture}
%         % };

%         % ------------------------------------------------
%         % 4. 装饰元素 (圆点)
%         % ------------------------------------------------
%         \fill[accentorange] ([xshift=3cm, yshift=-4cm]current page.north west) circle (0.15);
%         \fill[white] ([xshift=3.5cm, yshift=-3.5cm]current page.north west) circle (0.08);

%         % ------------------------------------------------
%         % 5. 底部页脚 (MOVED HERE) - 强制置于底部
%         % ------------------------------------------------
%         % 蓝色底条 (高度增加到 2.2cm 以防裁切)
%         \fill[techblue] (current page.south west) rectangle ([yshift=2.5cm]current page.south east);
        
%         % 页脚文字 (白色,稍微调高位置 yshift=0.9cm)
%         \node[anchor=south, white, yshift=0.5cm] at (current page.south) {
%             \begin{tabular}{c}
%                 \textbf{发布机构:} XX 研究院 \quad | \quad \textbf{联合发布:} XX 科技集团 \\
%                 \small 官方网站:www.leap-index.com \quad | \quad 联系电话:400-888-8888
%             \end{tabular}
%         };

%     \end{tikzpicture}

%     % --- 封面正文内容 ---
%     \vspace*{5cm} % 调整文字起始位置
    
%     \begin{center}
%         \textcolor{gray}{\fontsize{30}{40} \textbf{低空经济发展动态评估指数蓝皮书}} \\[0.5cm]
        
%         {
%             \fontsize{25}{35}\selectfont \textbf{\textcolor{deepspace}{基于领航模型(PILOT)的数据驱动洞察}} 
%         } \\[0.2cm]
        
%         \textcolor{techblue}{\Large \textsc{Performance} \textsc{Index} for \textsc{Low-altitude} \textsc{Operation} \&  \textsc{Technology}} \\[1.5cm]

% % PILOT-Performance Index for Low-altitude Operation & Technology
        
%         \begin{tcolorbox}[colback=skygradient, colframe=white, width=0.8\textwidth, arc=0mm, boxrule=0pt]
%             \centering \large \textcolor{deepspace}{``基于动态运行数据的五维评价体系,洞察万亿级蓝海新机遇''}
%         \end{tcolorbox}
%     \end{center}
    
%     % \vfill 不需要了,因为页脚已经固定绘制
    
% \end{titlepage}
\newgeometry{margin=0pt}

% ==========================================
%                 封面 (Cover)
% ==========================================
\begin{titlepage}
    \noindent
    \IfFileExists{images/cover.pdf}{%
        \includegraphics[width=\paperwidth, height=\paperheight]{images/cover.pdf}%
    }{%
        \IfFileExists{images/cover.png}{%
            \includegraphics[width=\paperwidth, height=\paperheight]{images/cover.png}%
        }{%
            % 备用方案...
            \begin{tikzpicture}[remember picture, overlay]
                \fill[blue] (current page.south west) rectangle (current page.north east);
                \node[white] at (current page.center) {Missing Cover File};
            \end{tikzpicture}%
        }%
    }%
\end{titlepage}
% 3. 恢复页边距,以免影响正文
\restoregeometry

\includepdf[pages=1]{images/cover.pdf}
% \includepdf[pages=1]{images/cover2.pdf}

% ==========================================
%                 第一章:概述
% ==========================================
\section*{摘要}

当前,全球低空经济正从技术验证迈向规模化商业应用的关键阶段,中国将其定位为发展新质生产力的重要引擎~\cite{item1}。然而,传统评估体系依赖静态、滞后的统计数据,难以精准、实时地刻画这一新兴业态的动态演进与真实效能,导致政策制定、资源配置与产业发展之间存在“数据时滞”与“决策盲区”~\cite{item2}。

为应对这一核心挑战,本蓝皮书提出了基于动态运行数据的低空经济发展五维评价模型-``领航''模型(PILOT) 。区别于国内现有的、以产业规模、政策文本和基础设施等静态或周期性统计数据为主的评价体系,本体系旨在构建一套基于全量化运行数据、高度模块化、可实时演算的城市低空经济发展度量衡,将低空经济视为一个持续运行的数字物理系统,其每一次飞行活动(架次、时长、航迹、主体)都是该系统健康状况最直接的“生命体征”~\cite{item3}。通过将高时空分辨率的实时飞行数据导入“规模、结构、时空、效能、创新”五维框架,实现从微观运行到宏观态势的精准洞察,推动了评估范式从“静态统计”向“动态感知”、从“宏观描述”向“微观诊断”的根本性转变~\cite{item4}。
本蓝皮书详细阐释了“领航”模型的理论基础、指数体系、计算逻辑及应用场景。本模型的核心价值在于其开源性与可操作性,任何城市均可依据本蓝皮书提供的标准化方法论,结合本地数据,生成定制化的评估报告,从而为精准施策、产业规划与商业决策提供“数据驾驶舱”式的支持,为培育高质量、可持续的低空经济生态贡献方法论基础。

% \subsection{什么是 LEAP?}
% LEAP 指数是一套基于**动态运行数据**的综合评价模型:

% \begin{itemize}
%     \item \textbf{L (Low-altitude Infrastructure) - 低空基建}:包括垂直起降点(Vertiports)、5G-A 通信网覆盖率。
%     \item \textbf{E (Economy \& Ecosystem) - 经济生态}:产业链企业的聚集度与投融资活跃度。
%     \item \textbf{A (Activity Level) - 运行活跃度}:无人机日均飞行架次与空域申请通过率。
%     \item \textbf{P (Policy \& Pilot) - 政策与试点}:地方法规的完善度与空域开放面积。
% \end{itemize}

% \subsection{Logo 替换指南}
% 在封面代码的第 \textbf{55-65} 行,您可以找到 Logo 的配置。

% \begin{tcolorbox}[colback=skygradient, colframe=techblue, title=操作步骤]
%     1. 准备一张透明背景的 PNG 图片(例如 \texttt{mylogo.png})。\\
%     2. 将代码中的占位符注释掉。\\
%     3. 启用 \texttt{includegraphics} 命令。
% \end{tcolorbox}


% ==========================================
%                 目录页
% ==========================================
\newpage
\tableofcontents
\newpage

\section{引言:迈向数据驱动的低空经济新时代}

低空经济,作为以各种有人驾驶和无人驾驶航空器的低空飞行活动为牵引,辐射带动相关领域融合发展的综合性经济形态,已成为全球航空业创新发展的前沿领域与战略制高点~\cite{item1}。在中国,发展低空经济被明确视为培育新质生产力的重要抓手,是推动产业升级、构建现代化交通体系、保障社会民生的重要引擎~\cite{item2}。
在全球低空经济从技术探索迈向规模化、商业化爆发的关键转折点上,一个根本性的挑战日益凸显:我们如何超越传统的定性描述与滞后的宏观统计,实时、精准、系统地度量这一新兴产业的真实发展脉搏?产业的蓬勃活力与决策的“数据迷雾”形成了鲜明对比。当前,对低空经济发展的评估主要依赖于企业数量、投资规模、基础设施等宏观、静态的“投入端”指标~\cite{item3},或聚焦于特定场景的案例分析~\cite{item4}。尽管这些研究为理解产业基础与政策环境提供了有益框架,但它们本质上仍是周期性的“快照”,难以实时捕捉并精准度量以高动态飞行活动为核心的真实经济运行状况。这种“静态统计”与“动态经济”之间的根本矛盾,导致决策者时常面临“看不清、判不准、跟不上”的困境~\cite{item5},也使得低空空域作为一种新型生产要素的巨大潜力难以科学评估和有效释放,从而难以从“可通达”的自然资源,高效转化为“可运营”的高价值经济资源。

为破解这一挑战,亟需一套以动态运行数据为基石的新型评估范式。构建一套以动态运行数据为核心、能够科学诊断低空经济系统健康状况的评估体系,不仅是学术研究的迫切需要,更是引导产业从“投入建设”迈向“高质量运营”、实现科学决策与精准施策不可或缺的基础设施~\cite{item6}。

秉承“创新智能技术,创造伟大企业,发展数字经济”的使命,坚持“科技擎天、产业立地”的理念,粤港澳大湾区数字经济研究院(IDEA研究院)低空经济分院(LASER Institute)认识到,实现低空经济高质量发展的核心路径,在于推动低空空域完成从“可通达”的自然资源,到“可计算”的数字资源,最终迈向“可运营”的高价值经济资源的转变~\cite{item7}。这一转变不仅需要SILAS(智能融合低空系统)以修建数字“空中之路”,更在于建立一套与之匹配的“价值评估系统”。唯有对空域中的动态运行活动进行科学度量,才能精准评估“设施网、空联网、航路网、服务网”的效能,回答产业“发展得如何”的根本问题~\cite{item8}。为此,我们发布本蓝皮书,系统阐述 以“领航”模型(PILOT-\textsc{Performance} \textsc{Index} for \textsc{Low-altitude} \textsc{Operation} \&  \textsc{Technology})为核心的一套基于多源动态运行数据的低空经济发展评估指数体系。

“领航”模型(PILOT)的核心革命在于,它将评估的指针从规划蓝图转向了真实飞行。本体系系统地将大规模、实时的飞行架次、航迹、时长、高度及关联主体数据作为评估的“生命体征”,通过 “规模与增长”、“结构与主体”、“时空特征”、“效能与质量”、“创新与融合”五个核心维度,该体系构建了一个能够持续感知产业脉搏、精准诊断系统健康度的“数据驾驶舱”。这一定位使“领航”模型(PILOT)与现有主要依赖统计数据的宏观指数形成了本质区别与关键互补。本蓝皮书旨在实现三个核心目标。第一,确立“以运行定义发展”的新评估基准。我们推动评估重心从“建造了什么”坚决转向“运行得如何”,致力于用真实经济活动的密度、效率和价值来客观衡量发展水平~\cite{item9}。第二,提供一套开源、可复用的方法论公共产品。我们完整公开“领航”模型的构建逻辑、指标体系与计算路径,旨在推动行业评估标准的共识,使其成为任何城市或区域均可便捷使用的“开箱即用”型分析工具。第三,赋能从经验决策到数据智能决策的治理升级~\cite{item10}。本体系不仅提供宏观排名,更能支持微观归因,帮助决策者精准识别基础设施短板、洞察市场结构变化、评估政策干预效果,最终实现资源的精准配置与产业的可持续发展。

我们深信,数据是驱动低空经济质变的核心引擎。《低空经济发展动态评估指数蓝皮书》的发布,是我们将空域“可计算”理念在产业评估领域的深度实践,也是我们践行“发展数字经济”使命、为行业贡献的一套开源、透明、可验证的公共方法论。我们期望将前沿智能技术与真实产业需求紧密结合,以数据智能驱动决策科学化,助力低空经济这一新质生产力在安全可控的轨道上实现规模化、高质量、可持续发展。我们期望与各界同仁一道,以“领航”模型为共同工具,拨开数据迷雾,以清晰的洞察引领低空经济驶向高质量、可持续的未来。

% \begin{enumerate}
%     \item \textbf{马太效应初显}:深圳、上海等头部城市占据了全行业 60\% 以上的低空飞行量。
%     \item \textbf{场景分化}:长三角地区更侧重于跨海物流,而珠三角地区在城市空中交通(UAM)方面走得更快。
% \end{enumerate}

\newpage
\section{发展背景:低空经济的崛起与核心特征}
\subsection{全球竞逐的战略新高地}
% 低空经济,通常指在垂直高度1000米以下、根据实际需求延伸至3000米以内的空域范围内,以各类有人驾驶和无人驾驶航空器的低空飞行活动为牵引,辐射带动相关领域融合发展的综合性经济形态~\cite{item11}。它不仅是通用航空业的延伸,更是融合了高端制造、人工智能、数字孪生等前沿技术的``新质生产力''典型代表~\cite{item12}。美国通过《先进空中交通(AAM)国家蓝图》等战略,系统规划城市空中交通(UAM)发展路径~\cite{item13};欧盟通过``可持续与智能交通战略'',大力推动无人机物流和空中出行服务~\cite{item14};国际民用航空组织(ICAO)也致力于制定全球协调框架,以安全集成无人机系统(UAS)~\cite{item1}。这些举措共同指向一个目标:抢占新一轮航空科技革命和产业变革的主导权。
%%%%%%%%%%%%%%%%%%%%%%%%%%%%%%%%jules
低空经济是指以低空空域(通常为1000米以下,可延伸至3000米)为依托,以各种有人及无人驾驶航空器飞行活动为牵引,带动相关领域融合发展的综合性经济形态~\cite{item11}。它集成了高端制造、人工智能与数字孪生等前沿技术,已成为全球新一轮科技革命与产业变革的必争之地。美国发布《先进空中交通(AAM)国家蓝图》以抢占UAM先机~\cite{item13},欧盟通过智能交通战略推动无人机物流~\cite{item14},ICAO亦加速制定全球协调框架~\cite{item1}。各国竞相布局,旨在掌控这一未来万亿级市场的战略主导权。

\subsection{中国发展的战略引擎}
% 在中国,低空经济被明确视为``战略性新兴产业''和``新增长引擎''~\cite{item15}。据相关规划,发展低空经济对于``构建现代产业体系、推动高质量发展、培育新增长极具有重大意义''~\cite{item16}。当前,中国低空经济正从试点探索迈向规范化、规模化发展的关键阶段,科学评估发展水平、识别短板、优化资源配置的需求日益迫切~\cite{item17}。

%%%%%%%%%%%%%%%%%%%%%%%%%%%%%%%%jules
在中国,低空经济被确立为``战略性新兴产业''与``新增长引擎''~\cite{item15},对于构建现代产业体系、培育新质生产力具有里程碑意义~\cite{item16}。当前,中国低空经济正从试点探索迈向规模化发展的关键期,迫切需要科学的评估体系以识别短板、优化配置,从而保障产业在``热潮''中实现高质量冷思考与稳健行进~\cite{item17}。

\subsection{产业内涵与核心特征}
% 低空经济并非传统通用航空的简单延伸,其核心特征决定了评估方式的革新必要性:
% \begin{itemize}
%     \item 运行主体的海量化与无人化:以无人机为代表的无人驾驶航空器正成为低空活动的主题,管理对象迈向``万架级''甚至``百万架级'',对空中交通管理系统提出了前所未有对容量和自动化管理挑战~\cite{item18}。
%     \item 应用场景的碎片化与融合化:场景高度分散且与实体经济各领域深度嵌套,陈献出强烈的融合经济特征~\cite{item19}。
%     \item 活动模式的动态化与网络化:飞行活动呈现高频次、短距离、强时效等特点,其经济价值与风险高度依赖于实时、动态变化的空域与交通流状态~\cite{item20}。
%     \item 技术体系的迭代化与集成化:电动化、智能化、网联化技术驱动航空器平台、动力系统、导航通信技术快速迭代,并与人工智能、物联网(IoT)、5G/6G等技术深度集成,持续重塑产业形态与运行模式~\cite{item4}。
% \end{itemize}
% 这些特征共同表明,低空经济的真实发展水平与健康状态,无法仅通过统计企业数量、投资规模或基础设施等静态``投入端''指标来准确反映。必须通过对产业核心活动——即大规模、高动态的实时飞行运行过程——进行持续、精准的度量,才能穿透表象,洞察其真实效能、内在结构与进化动力。 这正是本蓝皮书构建动态评估指数体系的逻辑起点与核心使命。

%%%%%%%%%%%%%%%%%jules
低空经济并非通用航空的简单延伸,其独特属性决定了评估逻辑的革新:
\begin{itemize}
    \item \textbf{主体海量化与无人化}:无人机成为绝对主力,管理对象从``千架级''跃升至``百万级'',对空域容量与自动化管理提出极限挑战~\cite{item18}。
    \item \textbf{场景碎片化与融合化}:应用深度嵌入物流、巡检、出行等实体经济末端,呈现高度分散且跨界融合的特征~\cite{item19}。
    \item \textbf{模式动态化与网络化}:高频次、短距离、强时效的飞行活动,使其价值与风险高度依赖于实时的空域状态与网络协同~\cite{item20}。
    \item \textbf{技术迭代化与集成化}:电动化、智能化技术驱动航空器快速迭代,与5G/6G、AI深度集成,持续重塑产业形态~\cite{item4}。
\end{itemize}
这些特征表明,低空经济的真实发展水平与健康状态,无法通过统计企业数量、投资规模或基础设施等静态``投入端''指标来准确反映。必须通过对产业核心活动——即大规模、高动态的实时飞行运行过程——进行持续、精准的度量,才能穿透表象,洞察其真实效能、内在结构与进化动力。 这正是本蓝皮书构建动态评估指数体系的逻辑起点与核心使命。
\newpage
\section{核心痛点:传统评估体系面临``五大失衡''}
% 在产业蓬勃发展的表象之下,一个关键的治理与决策难题日益凸显:我们如何科学、精准、实时地度量低空经济的真实发展水平? 现行依赖于传统统计和宏观指标的评估方式,与发展速度快、业态新、数字化程度高的低空经济之间,产生了深刻的``五大失衡'',形成了决策的``数据迷雾''。
% \begin{itemize}
%     \item ``静态数据''与``动态经济''的失衡:传统评估高度依赖企业注册数、固定资产投资额等静态、周期性统计数据~\cite{item4}。这些指标如同``快照'',只能反映某一时点的投入和存量,无法描述产业``实时运行''的活跃度、效率与质量。产业真实态势在静态数据中完全``失语''~\cite{item5}。
% \item ``宏观总量''与``微观结构''的失衡:现有报告多关注产业总体规模,但无法解答以下结构性关键问题:飞行活动是由少数企业垄断还是多元主体参与?主要集中于消费娱乐还是已渗透至物流、巡检等生产领域?在时空分布上是均衡发展还是热点拥堵与资源闲置并存?这种``只见森林、不见树木''的宏观视角,导致无法识别产业内部生态的健康度与发展的均衡性~\cite{item6}。
% \item ``事后统计''与``实时决策''的失衡:基于年报、统计公报的数据存在严重滞后期。用滞后数据指导快速发展的产业,如同``通过后视镜开车'',无法对基础设施不足、空域利用瓶颈、安全风险苗头等问题做出前瞻性预警和即时性响应。这种滞后性使得决策常处于被动应对状态,难以进行主动规划和精准调控~\cite{item21}。
% \item ``设施建设''与``运行效能''的失衡:易于度量基础设施的``硬投入'',却难以评估其``软利用''。建成的起降点利用率如何?规划的空域通道流量是否饱和?飞行任务的平均经济航程是多少?这些关乎发展质量与可持续性的``效能''问题,在传统框架下长期缺失。目前行业存在``基础设施超前投资''与``实际运营需求和利用率不明''之间的脱节风险,缺乏对设施利用效率和投资回报率的持续度量,可能导致资源错配~\cite{item22}。
% \item ``创新潜力''与``现实融合''的失衡:常用研发投入、专利数量衡量创新,但缺乏对技术成果在真实运行中渗透率与融合程度的实时观测~\cite{item10}。
% \end{itemize}
% 这五大失衡共同指向一个结论:必须推动评估范式发生根本性转变,从依赖周期性的、事后的、静态的统计报告,转向依靠连续的、实时的、动态的运行数据流进行分析。
%%%%%%%%%%%%%%%%%%%%%%%%%%%%%%%jules
在产业蓬勃发展的表象下,现行依赖传统统计的评估方式与数字化、高动态的低空经济之间存在深刻错位,导致决策层深陷``数据迷雾'',面临``五大失衡'':
\begin{itemize}
    \item \textbf{``静态数据''与``动态经济''的失衡}:传统指标(如企业数、投资额)仅是周期性``快照'',无法捕捉实时运行的活跃度与效率,导致产业真实态势在数据中``失语''~\cite{item5}。
    \item \textbf{``宏观总量''与``微观结构''的失衡}:关注总体规模而忽视结构健康。无法回答``谁在飞(垄断或多元)''、``飞什么(消费或生产)''、``在哪飞(拥堵或闲置)''等关键问题,难以识别生态的真实质量~\cite{item6}。
    \item \textbf{``事后统计''与``实时决策''的失衡}:滞后的年报数据如同``看后视镜开车'',无法对空域瓶颈、安全隐患等进行前瞻预警,致使决策被动,缺乏敏捷调控能力~\cite{item21}。
    \item \textbf{``设施建设''与``运行效能''的失衡}:重``硬投入''轻``软利用''。起降点与空域资源的实际利用率往往成谜,缺乏对投资回报的持续度量,易引发资源错配与无效建设风险~\cite{item22}。
    \item \textbf{``创新潜力''与``现实融合''的失衡}:常以专利数衡量创新,却缺乏对技术在真实场景中渗透率与融合度的观测,难以判断技术转化的实际成效~\cite{item10}。
\end{itemize}
这五大失衡表明,评估范式必须发生根本性变革:从依赖周期性、静态的统计报告,转向基于连续、实时运行数据的动态诊断。


\newpage
\section{范式革新:引入动态数据驱动的领航模型}
% 为穿透``数据迷雾'',本蓝皮书提出全新的 低空经济发展动态评估指数体系,其核心是名为``领航''(PILOT)的五维评价模型 。这一范式革新的核心在于确立一个新基准:低空经济的价值最终通过安全、高效、大规模的飞行活动来实现。 因此,持续产生并汇聚的飞行动态数据是衡量其发展水平最直接、最客观、最及时的``金标准''~\cite{item5}。通过构建一个多维度的动态数据评价体系,我们可以将产业的``抽象潜力''转化为``具象的、可度量的运行现实''。
%%%% jules
为穿透``数据迷雾'',本蓝皮书提出全新的 低空经济发展动态评估指数体系,其核心是名为``领航''(PILOT)的五维评价模型。这一范式革新的核心在于确立一个新基准:低空经济的价值最终通过安全、高效、大规模的飞行活动来实现。 因此,持续产生并汇聚的飞行动态数据是衡量其发展水平最直接、最客观、最及时的``金标准''~\cite{item5}。通过构建一个多维度的动态数据评价体系,我们可以将产业的``抽象潜力''转化为``具象的、可度量的运行现实''。

\subsection{核心理念}
% 低空经济的价值最终通过航空器的实际运行来创造与体现。每一次飞行都是一次经济活动的发生、一次生产要素的流动、一次技术能力的验证。因此,对飞行运行本身进行全维度、全链条的深度分析,是评估低空经济发展最直接、最真实、最动态的途径。本评估体系将每一次飞行任务的架次、航迹、时长、高度、速度及关联主体信息,都视为产业活力的最小单元和经济价值的直接体现。评估焦点从传统的``建造了多少''(投入),根本性转向``运行得怎样''(产出与效能)~\cite{item6}。因此,我们构建了一条清晰的``数据-指标-维度-洞察''转化管道,确保每一个评估结论都根植于可验证的数据事实。
%%%%%%%%%%%%%%%jules
低空经济的价值最终通过航空器的实际运行来创造与体现。每一次飞行都是一次经济活动的发生、一次生产要素的流动、一次技术能力的验证。因此,对飞行运行本身进行全维度、全链条的深度分析,是评估低空经济发展最直接、最真实、最动态的途径。本评估体系将每一次飞行任务的架次、航迹、时长、高度、速度及关联主体信息,都视为产业活力的最小单元和经济价值的直接体现。评估焦点从传统的``建造了多少''(投入),根本性转向``运行得怎样''(产出与效能)~\cite{item6}。因此,我们构建了一条清晰的``数据-指标-维度-洞察''转化管道,确保每一个评估结论都根植于可验证的数据事实。
\begin{itemize}
    \item 数据驱动:以全量、实时、真实的低空飞行动态数据为客观输入,确保评估的即时性与真实性~\cite{item24}。
\item 系统视角:从规模、结构、时空、效能、创新五个相互关联的维度解构低空经济复杂系统,避免单一指标的片面性~\cite{item25}。
% \item 价值导向:不仅衡量``有没有''、``多不多'',更重点评估``好不好''、``优不优''、``新不新'',引导产业走向高质量发展~\cite{item26}。
\item 价值导向:不仅衡量``有没有''、``多不多'',更重点评估``好不好''、``优不优''、``新不新'',引导产业走向高质量发展~\cite{item26}。
\item 开源透明:倡导开放的方法论与可复现的计算逻辑,推动行业评估标准共建~\cite{item27}。
\end{itemize}
\subsection{体系价值}
% 本评估体系的根本价值在于,它通过核心的``领航''模型(PILOT)构建了一个能够系统解构低空经济复杂性的标准化评估框架,将多维度的运行活动转化为可量化、可比较、可解释的决策知识,为产业从规模化扩张向高质量发展转型提供了关键的基础设施~\cite{item6}。
本评估体系的根本价值在于,它通过核心的``领航''模型(PILOT)构建了一个能够系统解构低空经济复杂性的标准化评估框架,将多维度的运行活动转化为可量化、可比较、可解释的决策知识,为产业从规模化扩张向高质量发展转型提供了关键的基础设施~\cite{item6}。

对政府与监管方而言,它是科学决策的系统工具:
\begin{itemize}
    % \item 实现发展质量的精准诊断:模型超越总量统计,通过``效能与质量''、``结构与主体''等维度,精准识别区域发展在运行效率、市场健康度、应用结构等方面的真实短板,使产业扶持政策从``大水漫灌''转向``精准滴灌''~\cite{item28}。
    \item 实现发展质量的精准诊断:模型超越总量统计,通过``效能与质量''、``结构与主体''等维度,精准识别区域发展在运行效率、市场健康度、应用结构等方面的真实短板,使产业扶持政策从``大水漫灌''转向``精准滴灌''~\cite{item28}。
    % \item 支撑空域与资源的精细化规划:基于``时空特征''与``效能''指数形成的分析结论,能够为航路网络优化、起降场站布局提供量化的需求依据,推动基础设施投资从``按需建设''迈向``按效规划'',提升公共资源使用效益~\cite{item29}。
    \item 支撑空域与资源的精细化规划:基于``时空特征''与``效能''指数形成的分析结论,能够为航路网络优化、起降场站布局提供量化的需求依据,推动基础设施投资从``按需建设''迈向``按效规划'',提升公共资源使用效益~\cite{item29}。
    \item 建立跨区域对标与协同发展的基准:统一的五维指数体系使不同城市、省份乃至国家间的低空经济发展水平具备可比性,有助于识别差距、分享最佳实践,并为区域协同规划提供共同的评估语言~\cite{item30}。
\end{itemize}

对产业与市场主体而言,它是战略导航的量化仪表:
\begin{itemize}
    \item 提供深度市场洞察与竞争分析:企业可通过指数解构,清晰把握不同区域的场景成熟度、市场竞争格局(如集中度指数)及技术渗透趋势,为业务布局、产品定位提供超越直觉的数据支撑~\cite{item31}。
\item 确立内部运营优化的标杆:模型将行业整体运行效能予以量化呈现,为企业对标行业平均或领先水平、发现自身在任务规划、资产利用等方面的提升空间提供了客观标尺~\cite{item32}。
% \item 辅助投资价值发现与风险管理:投资者可依据``创新与融合''及``结构''指数,系统性评估赛道的前沿性、生态健康度与企业梯队,从而更有效地甄别具有长期潜力的投资对象,管理投资组合风险~\cite{item33}。
\item 辅助投资价值发现与风险管理:投资者可依据``创新与融合''及``结构''指数,系统性评估赛道的前沿性、生态健康度与企业梯队,从而更有效地甄别具有长期潜力的投资对象,管理投资组合风险~\cite{item33}。
\end{itemize}

对行业与研究界而言,它是推动共识的方法论基础:
\begin{itemize}
    % \item ``领航''模型(PILOT)提供了一套开源、透明、可验证的评估方法论。其系统性的维度设计和标准化的指标构建,旨在推动学术界、咨询界及行业协会就``如何科学衡量低空经济发展''形成共识,促进评估标准的统一与数据的互联互通,为构建健康的产业生态奠定认知基础~\cite{item34}。
    \item ``领航''模型(PILOT)提供了一套开源、透明、可验证的评估方法论。其系统性的维度设计和标准化的指标构建,旨在推动学术界、咨询界及行业协会就``如何科学衡量低空经济发展''形成共识,促进评估标准的统一与数据的互联互通,为构建健康的产业生态奠定认知基础~\cite{item34}。
\end{itemize}

% 总而言之,以``领航''模型(PILOT)为核心的基于多源动态运行数据的低空经济发展评估指数体系是连接低空经济``动态现实''与``科学决策''的桥梁。它通过将海量运行数据转化为系统的洞察,旨在终结``凭经验决策''与``用静态规划动态''的时代,推动整个产业迈向基于精准感知和数据智能的新阶段。
总而言之,以``领航''模型(PILOT)为核心的基于多源动态运行数据的低空经济发展评估指数体系是连接低空经济``动态现实''与``科学决策''的桥梁。它通过将海量运行数据转化为系统的洞察,旨在终结``凭经验决策''与``用静态规划动态''的时代,推动整个产业迈向基于精准感知和数据智能的新阶段。


\newpage
\section{从数据到指数的计算路径}
本指数体系的构建遵循一套标准化、可复现的计算路径,实现了从海量多源异构数据到宏观决策指数的科学转化。本章详细阐述从基础指标到维度子指数、维度指数,最终合成综合指数的完整算法逻辑与实施步骤。

\subsection{总体计算路径:四层聚合流程}
指数计算遵循自下而上的“四层聚合”逻辑,确保每一层级的数值均具有明确的物理意义与决策指向(见~\cref{fig4}):
\begin{figure}[H]
    \centering
    \includegraphics[width=0.9\textwidth]{images/图3.png}
    \caption{四层聚合流程}
    \label{fig4}
\end{figure}

\subsection{核心计算方法详述}

\subsubsection{基础指标计算}
此阶段将原始数据清洗、加工为可直接度量的基础指标值。
\begin{table}[H]
    \centering
    \small
    \begin{tabular}{|l|p{8cm}|}
\hline 输入 & 清洗后的飞行轨迹数据、飞行计划数据、航空器注册信息及用户主数据。 \\
\hline 处理 &
\textbf{时空聚合}:按预设时空单元(如日/月/年、城市/网格)对飞行事件进行统计聚合。 \newline
\textbf{业务逻辑计算}:依据指标定义执行特定运算(如去重、比率计算)。\\
\hline 输出 & 构成“维度子指数”计算基础的数十个基础指标值。 \\
\hline
\end{tabular}
    \caption{基础指标计算流程}
    \label{tab8}
\end{table}

\subsubsection{权重确定:主客观混合赋权}
为兼顾数据客观性与战略导向性,本模型采用\textbf{主客观混合赋权法}~\cite{item44}:
\begin{itemize}
    \item \textbf{客观赋权(熵权法 EWM)}:应用于\textbf{基础指标对子指数}的权重确定。依据指标数据的离散程度(熵值)自动分配权重,数据差异越大,信息量越大,权重越高,确保由数据本身的变异性驱动评价~\cite{item46}。
    \item \textbf{主观赋权(层次分析法 AHP)}:应用于\textbf{子指数对维度指数}、\textbf{维度指数对综合指数}的权重确定。通过专家打分构建判断矩阵,将产业发展阶段特征与区域战略重点转化为定量权重,体现评价的战略引导功能~\cite{item45}。
\end{itemize}

\textbf{熵权法计算核心步骤:}
假设有 $n$ 个样本和 $m$ 个指标,构建原始矩阵 $\mathbf{X}=(x_{ij})_{n \times m}$。
1. \textbf{标准化}:消除量纲差异。
   正向指标:$r_{ij} = \frac{x_{ij} - \min(x_{j})}{\max(x_{j}) - \min(x_{j})}$;
   负向指标:$r_{ij} = \frac{\max(x_{j}) - x_{ij}}{\max(x_{j}) - \min(x_{j})}$。
2. \textbf{计算熵值}:$e_j = -k \sum_{i=1}^{n} p_{ij} \ln(p_{ij})$,其中 $p_{ij} = r_{ij}/\sum r_{ij}$,$k = 1/\ln(n)$。
3. \textbf{确定权重}:$w_j = \frac{1-e_j}{\sum (1-e_j)}$。差异系数 $1-e_j$ 越大,权重越高。

\subsubsection{指数合成方法}
采用线性加权求和法进行逐层聚合,确保计算透明与结果可解释:

\textbf{第一层:维度子指数合成(基于客观权重)}
在同一维度内,利用熵权法计算基础指标权重 $w_j$,合成子指数 $S_{ikl}$(第 $i$ 样本第 $k$ 维度第 $l$ 子指数):
$$
S_{i k l}=\sum_{j=1}^{m_{k l}}\left(w_j \times r_{i j}\right) \times 100
$$

\textbf{第二层:维度指数合成(基于战略权重)}
利用AHP确定各子指数的战略权重 $W_{kl}$,合成维度指数 $D_{ik}$:
$$
D_{i k}=\sum_{l=1}^{L_k}\left(W_{k l} \times S_{i k l}\right)
$$

\textbf{第三层:综合指数合成(基于顶层权重)}
基于AHP确定的维度权重 $V_k$,合成最终的低空综合繁荣指数(LA-CPI):
$$
C I_i=\sum_{k=1}^K\left(V_k \times D_{i k}\right)
$$

\subsection{动态追踪与更新机制}
为确保指数的时效性与长期可比性,建立全生命周期的动态管理机制:
\begin{itemize}
    \item \textbf{多频次发布}:
    \begin{itemize}
        \item \textbf{实时/周/月}:针对流量、热度等高频指标,通过仪表盘实时呈现。
        \item \textbf{年度}:发布权威综合报告(LA-CPI),进行深度归因与趋势研判。
    \end{itemize}
    \item \textbf{动态基准期管理}:设定首个完整评估年度为基准期(100点)。每3-5年或遇产业重大结构性变革时,进行基准期复审与回溯调整,确保长周期趋势的可比性。
    \item \textbf{权重动态校准}:
    \begin{itemize}
        \item \textbf{客观权重}:随每期数据自动更新,实时反映指标波动。
        \item \textbf{主观权重}:每1-2年组织专家复审,依据产业阶段变化(如从基建期转向应用期)动态调整战略权重。
    \end{itemize}
    \item \textbf{版本控制}:对指标体系、算法与权重的任何调整均实行严格的版本管理,发布《变更说明》,确保指数的公信力与透明度。
\end{itemize}
这一机制使“领航”模型不仅是静态的标尺,更是能够自我进化、持续适应产业发展的动态监测系统。


\newpage
\section{指数解读与城市画像:从数据到洞察}

\subsection{指数解读:三层诊断法}
本体系的终极价值在于将复杂的量化结果转化为清晰的决策洞察。我们提出``综合定位-维度诊断-微观归因''的三层解读路径,确保从宏观战略到微观行动的无缝衔接~\cite{item42}。

\begin{itemize}
    \item \textbf{第一层:综合定位(看总分)}。依据LA-CPI总分与排名,明确城市在低空经济版图中的生态位(如``领跑者''、``追赶者''或``起步者'')。
    \item \textbf{第二层:维度诊断(看雷达图)}。通过五维雷达图的形态识别结构性优劣势:
    \begin{itemize}
        \item \textbf{均衡性}:完美的正五边形极为罕见,健康的产业往往在``规模''与``效能''间保持动态平衡。
        \item \textbf{短板识别}:若``效能''或``创新''维度显著凹陷,提示增长模式粗放或缺乏后劲;若``结构''维度得分低,则需警惕市场垄断或应用场景单一。
    \end{itemize}
    \item \textbf{第三层:微观归因(看指标)}。针对问题维度,下钻至子指数与基础指标寻找根源。例如,``效能指数''偏低,究竟是``单机利用率''不足,还是``空域流转''不畅?从而实现精准施策。
\end{itemize}

\subsection{发展模式辨识:从评估到定位}
超越简单排名,本体系更注重辨识城市发展的内在驱动力。基于五维指数特征,我们定义了四种典型的低空经济发展模式,为城市制定差异化战略提供参照。

\subsubsection{四种典型发展模式}
\begin{table}[H]
    \centering
    \footnotesize
    \rotatebox{90}{
\begin{tabular}{|p{2cm}|p{4cm}|p{4cm}|p{4cm}|p{4cm}|}
\hline \rowcolor{skygradient!100} \multicolumn{1}{|c|}{模式类型} & \multicolumn{1}{|c|}{核心驱动力} & \multicolumn{1}{|c|}{关键指数特征} & \multicolumn{1}{|c|}{典型雷达图} & \multicolumn{1}{|c|}{战略建议} \\
\hline
\textbf{制造驱动型} & \textbf{``研产牵引''}:以航空器整机及核心部件研发制造为主导,飞行活动多为试飞验证,商业化运营尚处于培育期。 &
\textbf{强}:``规模''、``创新''(试飞活跃)\newline
\textbf{弱}:``效能''、``结构''(商业场景少)
 & \textbf{``哑铃型''}:规模与创新两端突出,中间运营环节凹陷,反映``研产强、应用弱''的断层。 & 推动``从造飞向用''。利用制造优势开放试飞场景,引入运营企业,加速将技术优势转化为市场服务优势。 \\
\hline
\textbf{场景深化型} & \textbf{``单点突破''}:依托特定高价值场景(如海岛物流、景区观光)形成商业闭环,以点带面驱动产业发展。 & \textbf{强}:``效能''、``融合''(特定场景效率高)\newline
\textbf{稳}:``规模''(需求驱动,增长稳健) & \textbf{``钻石型''}:效能与融合维度突出,形态扎实。 & 聚焦``立法、定标、拓面''。将标杆场景经验转化为标准规范,推动从``盆景''向``风景''的规模化复制。 \\
\hline
\textbf{基建引领型} & \textbf{``筑巢引凤''}:超前布局起降设施、通信导航与数据平台,以优质公共服务吸引主体集聚。 & \textbf{强}:``时空''(覆盖广、网络化)\newline
\textbf{潜}:``结构''(正处于主体导入期) & \textbf{``金字塔型''}:时空维度构成宽广底座,承载力强,等待上层应用爆发。 & 强化``开放共享与服务运营''。推动设施互联互通,发展机库运维、数据服务等后市场,从``通道经济''升级为``平台经济''。 \\
\hline
\textbf{生态培育型} & \textbf{``内生繁荣''}:营商环境优越,中小企业活跃,应用场景多元,呈现自下而上的内生增长特征。 & \textbf{强}:``结构''(主体多元、梯队健康)\newline
\textbf{均}:各维度无明显短板 & \textbf{``饱满圆形''}:五维均衡,抗风险能力强,是成熟生态的典型体现。 & 维护``公平竞争与创新激励''。提供普惠性公共服务,降低准入门槛,持续激发微观主体的创新活力。 \\
\hline
\end{tabular}}
    \caption{低空经济典型发展模式画像}
    \label{table8}
\end{table}

\subsubsection{模式演进与战略应用}
\begin{itemize}
    \item \textbf{混合与过渡}:现实中多数城市呈现混合特征。识别主导模式(如``基建搭台、制造唱戏'')有助于厘清当前的核心矛盾。
    \item \textbf{动态演进}:城市发展路径往往遵循``制造/基建驱动 $\rightarrow$ 场景深化 $\rightarrow$ 生态培育''的螺旋上升规律。通过年度雷达图对比,可清晰描绘转型轨迹。
    \item \textbf{精准对标}:避免盲目``唯总量论''。制造型城市应重点对标如何补齐应用短板,而基建型城市应关注设施利用率的提升。本体系为``同类项对标''提供了科学依据。
\end{itemize}
通过模式辨识,本指数从单纯的``体检表''升维为``战略罗盘'',帮助城市认清``现在在哪里'',明确``应向何处去''。


\newpage
\section{低空经济发展指数动态评估模型的应用}
本体系的核心使命在于将数据转化为洞察,进而转化为行动。本章为政府、企业及投资机构提供一套具体的决策应用指南。

\begin{itemize}
    \item \textbf{政府与监管方:精准治理的“仪表盘”}
    \begin{itemize}
        \item \textbf{政策效能评估}:定量追踪政策出台前后关键指数(如“夜间经济指数”)的边际变化,实现从“定性评价”向“数据实证”的转变。
        \item \textbf{资源精准配置}:依据“区域均衡指数”和“时空分布”热力图,精准识别基础设施盲区与空域瓶颈,指导财政资金投向急需之处。
        \item \textbf{系统风险预警}:实时监控“规模”高增与“效能/合规”指标背离的异常信号,防范“虚假繁荣”与安全隐患。
    \end{itemize}

    \item \textbf{运营与制造企业:市场拓展的“导航仪”}
    \begin{itemize}
        \item \textbf{高潜区域锁定}:利用“结构”与“需求”维度数据,识别竞争蓝海与业务高热区,优化市场布局。
        \item \textbf{运营效能对标}:以行业平均的“单机作业效能”为基准,诊断自身资产利用效率,发掘降本增效空间。
        \item \textbf{生态位匹配}:参考目标城市的“发展模式”画像,判断其产业土壤是否契合自身业务基因(如试飞需求 vs. 商业运营需求)。
    \end{itemize}

    \item \textbf{投资与研究机构:价值发现的“透视镜”}
    \begin{itemize}
        \item \textbf{赛道优选}:结合“综合指数”与“创新融合”维度,优先布局具备内生增长动力与技术溢出效应的区域。
        \item \textbf{去伪存真}:通过“活跃运力”与“真实飞行时长”等硬核指标,穿透PPT概念,识别具备真实业务壁垒的“领航”企业。
        \item \textbf{趋势研判}:基于长周期指数演变,捕捉从“制造驱动”向“服务驱动”转型的关键拐点。
    \end{itemize}
\end{itemize}

\newpage
\section{局限性与未来展望}
作为基于动态运行数据的首创性探索,本模型在提供精准洞察的同时,亦面临客观局限,这些局限指明了未来的迭代方向。

\subsection{当前局限}
\begin{itemize}
    \item \textbf{数据完备性挑战}:评估精度高度依赖数据的全量接入。当前部分区域低空数据尚未完全打通,可能导致局部评估偏差。
    \item \textbf{因果解释的复杂性}:指数能精准揭示“是什么”和“在哪里”,但对“为什么”(如空域闲置的深层体制原因)的解释仍需结合定性调研。
    \item \textbf{外部冲击的非线性}:极端天气或突发管控等不可抗力可能导致指数短期剧烈波动,需在长期趋势分析中予以剔除或平滑。
\end{itemize}

\subsection{未来演进}
面对产业的快速迭代,“领航”模型(PILOT)将向更智能、更融合的方向演进:
\begin{itemize}
    \item \textbf{多源融合}:从单一“运行数据”驱动,迈向“运行+经济+地理+社会”多模态数据融合驱动。
    \item \textbf{预测模拟}:引入AI大模型,从“监测现状”升级为“推演未来”,支持政策沙箱模拟。
    \item \textbf{实时感知}:推动评估频次从“月度/年度”向“天/小时”级实时动态监测跃升。
    \item \textbf{开源共建}:推动核心指标定义的行业标准化,构建开放共享的评估生态。
\end{itemize}

\newpage
\section{结论与倡议}
本蓝皮书提出的“领航”模型(PILOT),标志着低空经济评估范式的一次根本性跨越:从静态统计走向动态感知,从宏观概括走向微观诊断,从衡量“建设投入”走向评估“真实效能”。它首次将流动的低空飞行数据确立为衡量产业价值的核心标尺。

我们深知,标准的确立需要行业的共同智慧。为此,我们倡议:
\begin{itemize}
    \item \textbf{数据开放与互通}:打破数据孤岛,推动建立城市间、政企间可信的数据共享机制。
    \item \textbf{标准共建与迭代}:以开源精神共同维护指标体系,使其随产业实践动态进化。
    \item \textbf{价值导向与回归}:坚持“以运行论英雄”,引导产业资源从“跑马圈地”回归到创造真实经济社会价值的轨道上来。
\end{itemize}

让我们携手以数据为眼,洞察低空经济的真实脉动,共同引领这一万亿级蓝海驶向高质量、可持续的未来。


\newpage
\section*{附录: 维度子指数计算表}

\begin{table}[H]
    \centering
    \small
    \rotatebox{90}{
    \begin{tabular}{|p{1cm}|p{2cm}|p{2.5cm}|p{2.5cm}|p{4cm}|p{2cm}|}
\hline \rowcolor{skygradient!100} \multicolumn{1}{|l|}{序号} & \multicolumn{1}{|c|}{指数名称} & \multicolumn{1}{|c|}{定义} & \multicolumn{1}{|c|}{作用} & \multicolumn{1}{|c|}{计算公式} & \multicolumn{1}{|c|}{数据源} \\
\hline 1 & 低空交通流量指数 & 衡量区域低空飞行活动的绝对规模与市场容量。 & 衡量产业基本盘的核心指标,直观反映城市``繁忙度''。 & $\displaystyle \frac{F_t}{F_0} \times 100$ \newline
\textit{$F_t$: 当期架次, $F_0$: 基期架次} & 每日飞行架次统计 \\
\hline 2 & 低空作业强度指数 & 衡量单位空域面积或单位时间内的飞行活动密度。 & 反映空域资源利用压力与业务负荷。 &  $\displaystyle \left[w_1 \frac{T_t}{T_0} + w_2 \frac{D_t}{D_0}\right] \times 100$ \newline
\textit{$T_t$: 当期时长, $D_t$: 当期里程} & 年合计飞行时长、里程 \\
\hline 3 & 活跃运力规模指数 & 衡量实际投入运营的航空器及主体数量。 & 反映供给侧投入程度与参与广度。 & $N_a$ \newline
\textit{$N_a$: 活跃航空器SN码数量} & 活跃SN数 \\
\hline 4 & 增长动能指数 & 衡量飞行规模扩张速度与趋势。 & 判断市场处于爆发期、稳定期或瓶颈期。 & $\displaystyle R_m = \frac{F_m - F_{m-1}}{F_{m-1}} \times 100\%$ \newline
\textit{$R_m$: 环比增长率} & 月度飞行架次 \\
\hline 5 & 市场集中度指数 & 衡量头部企业对市场的控制力。 & 判断市场竞争格局(垄断/竞争)与风险。 & $\displaystyle CR_{50} = \frac{\sum_{i=1}^{50} F_i}{F_{\text{total}}} \times 100$ \newline
\textit{$CR_{50}$: 前50企业份额} & 企业飞行架次 \\
\hline
\end{tabular}}
    \caption{低空经济指数计算公式汇总}
    \label{tab:index-formulas}
\end{table}

\begin{table}[H]
    \centering
    \small
    \rotatebox{90}{
    \begin{tabular}{|p{1cm}|p{2cm}|p{2.5cm}|p{2.5cm}|p{2.5cm}|p{2cm}|}
\hline \rowcolor{skygradient!100} \multicolumn{1}{|c|}{序号} & \multicolumn{1}{|c|}{指数名称} & \multicolumn{1}{|c|}{定义} & \multicolumn{1}{|c|}{作用} & \multicolumn{1}{|c|}{计算公式} & \multicolumn{1}{|c|}{数据源} \\
\hline 6 & 商业化成熟指数 & 衡量飞行活动中生产性、商业化应用的占比。 & 反映产业``自我造血''能力与真实价值。 & (企业用户架次/总架次) $\times 100$ & 用户类型统计 \\
\hline 7 & 机型生态多元指数 & 衡量运营航空器型号的丰富性与均衡性。 & 反映技术路线丰富度与供应链韧性。 & $1-\Sigma\left(\mathrm{P_i}^2\right)$ \newline
\textit{辛普森多样性指数} & 机型飞行架次 \\
\hline 8 & 区域发展均衡指数 & 衡量飞行活动在地理单元分布的均衡性。 & 识别热点与盲区,指导基础设施布局。 & $1 - G$ \newline
\textit{G: 基尼系数} & 行政区飞行架次 \\
\hline 9 & 全时段运行指数 & 衡量飞行活动在24小时内的分布离散度。 & 评估全天候运行能力与夜间经济潜力。 & $-\Sigma\left(\mathrm{P_i} \cdot \ln \mathrm{P_i}\right)$ \newline
\textit{信息熵} & 时段飞行架次 \\
\hline 10 & 季候稳定性指数 & 衡量飞行活动受季节、月份影响的波动程度。 & 判断产业是否实现全年常态化运行。 & $1 - \frac{\sigma}{\mu}$ \newline
\textit{$\sigma$: 标准差, $\mu$: 均值} & 月度飞行架次 \\
\hline
\end{tabular}}
    \caption{低空经济指数计算公式汇总(续)}
    \label{tab:index-formulas-2}
\end{table}

\begin{table}[H]
    \centering
    \small
    \rotatebox{90}{
    \begin{tabular}{|p{1cm}|p{2cm}|p{6cm}|p{2cm}|p{5cm}|p{2cm}|}
\hline \rowcolor{skygradient!100} \multicolumn{1}{|c|}{序号} & \multicolumn{1}{|c|}{指数名称} & \multicolumn{1}{|c|}{定义} & \multicolumn{1}{|c|}{作用} & \multicolumn{1}{|c|}{计算公式} & \multicolumn{1}{|c|}{数据源} \\
\hline 11 & 网络化枢纽指数 & 综合计算起降点的连接度、加权流量和中介中心性。 & 识别关键枢纽,指导基建投资。 & $NHI = \alpha \cdot DC + \beta \cdot WF + \gamma \cdot BC$ & 航线频次 \\
\hline 12 & 单机作业效能指数 & 衡量平均每架航空器的年度使用频率。 & 评估资产运营效率。 & $\displaystyle E = \frac{N_{\text{总}}}{M_{\text{活跃}}}$ & 总架次/活跃SN数 \\
\hline 13 & 长航时任务占比指数 & 衡量长航时($>T$分钟)高价值任务的比重。 & 反映执行复杂任务的技术可靠性。 & $\displaystyle L = \frac{N_{>T}}{N_{\text{总}}} \times 100\%$ & 时长区间架次 \\
\hline 14 & 广域覆盖能力指数 & 衡量飞行活动覆盖的地理范围广度。 & 评估服务网络的覆盖通达性。 & $\displaystyle C = \sum \left(p_i \times d_i\right)$ & 里程区间架次 \\
\hline
\end{tabular}}
    \caption{低空经济指数计算公式汇总(续二)}
    \label{tab:index-formulas-3}
\end{table}

\begin{table}[H]
    \centering
    \small
    \rotatebox{90}{
    \begin{tabular}{|p{1cm}|p{2cm}|p{4cm}|p{3cm}|p{2cm}|p{2cm}|}
\hline \rowcolor{skygradient!100} \multicolumn{1}{|c|}{序号} & \multicolumn{1}{|c|}{指数名称} & \multicolumn{1}{|c|}{定义} & \multicolumn{1}{|c|}{作用} & \multicolumn{1}{|c|}{计算公式} & \multicolumn{1}{|c|}{数据源}  \\
\hline 15 & 任务完成质量指数 & 衡量飞行任务按计划执行的完整性。 & 监控运行安全与服务质量。 & (有效架次/计划架次) $\times 100$ & 飞行计划数据  \\
\hline 16 & 城市微循环渗透指数 & 衡量航空器在楼宇、社区等末端场景的应用深度。 & 反映低空经济与智慧城市的融合度。 & (跨区架次/总架次) $\times$ 跨区对数 & 跨区飞行架次  \\
\hline 17 & 立体空域利用效能指数 & 衡量不同高度层差异化利用的效率。 & 评估空域精细化管理水平。 & $-\Sigma\left(P_i \ln P_i\right)$ & 高度层时长  \\
\hline 18 & 生产/消费属性指数 & 对比工作日与周末活跃度,判断驱动属性。 & 判断产业主要驱动力(生产vs消费)。 & 工作日均架次 / 周末日均架次 & 每日架次 \\
\hline
\end{tabular}}
    \caption{低空经济指数计算公式汇总(续三)}
    \label{tab:my_label}
\end{table}

\begin{table}[H]
    \centering
    \small
    \rotatebox{90}{
    \begin{tabular}{|p{1cm}|p{2cm}|p{4cm}|p{2cm}|p{5cm}|p{2cm}|}
\hline \rowcolor{skygradient!100} \multicolumn{1}{|c|}{序号} & \multicolumn{1}{|c|}{指数名称} & \multicolumn{1}{|c|}{定义} & \multicolumn{1}{|c|}{作用} & \multicolumn{1}{|c|}{计算公式} & \multicolumn{1}{|c|}{数据源} \\
\hline 19 & 低空夜间经济指数 & 衡量夜间飞行活跃度。 & 挖掘``时间增量''价值,评估全天候能力。 & $\displaystyle N = \frac{N_{\text{夜}}}{N_{\text{总}}} \times 100\%$ & 时段飞行架次 \\
\hline 20 & 头部企业``领航''指数 & 衡量头部企业在技术与模式上的引领作用。 & 评估头部企业在高端任务中的领导力。 & $\displaystyle L = \frac{N_{\text{Top5}}}{N_{\text{总}}} \times 100\%$ & Top5单位数据 \\
\hline
\end{tabular}}
    \caption{低空经济指数计算公式汇总(续四)}
    \label{tab:index-formulas-5}
\end{table}


\newpage
\section*{附录: 示例}
\includepdf[pages=1-21]{docs/example.pdf}

\bibliographystyle{plain}
\bibliography{ref}

\end{document}